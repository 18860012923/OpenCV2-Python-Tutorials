


% Header, overrides base

    % Make sure that the sphinx doc style knows who it inherits from.
    \def\sphinxdocclass{article}

    % Declare the document class
    \documentclass[letterpaper,10pt,english]{/usr/lib/python2.7/site-packages/sphinx/texinputs/sphinxhowto}

    % Imports
    \usepackage[utf8]{inputenc}
    \DeclareUnicodeCharacter{00A0}{\\nobreakspace}
    \usepackage[T1]{fontenc}
    \usepackage{babel}
    \usepackage{times}
    \usepackage{import}
    \usepackage[Bjarne]{/usr/lib/python2.7/site-packages/sphinx/texinputs/fncychap}
    \usepackage{longtable}
    \usepackage{/usr/lib/python2.7/site-packages/sphinx/texinputs/sphinx}
    \usepackage{multirow}

    \usepackage{amsmath}
    \usepackage{amssymb}
    \usepackage{ucs}
    \usepackage{enumerate}

    % Used to make the Input/Output rules follow around the contents.
    \usepackage{needspace}

    % Pygments requirements
    \usepackage{fancyvrb}
    \usepackage{color}

    % Needed to box output/input
    \usepackage{tikz}
        \usetikzlibrary{calc,arrows,shadows}
    \usepackage[framemethod=tikz]{mdframed}

    \usepackage{alltt}

    % Used to load and display graphics
    \usepackage{graphicx}
    \graphicspath{ {figs/} }
    \usepackage[Export]{adjustbox} % To resize


    % For formatting output while also word wrapping.
    \usepackage{listings}
    \lstset{breaklines=true}
    \lstset{basicstyle=\small\ttfamily}
    \def\smaller{\fontsize{9.5pt}{9.5pt}\selectfont}

    %Pygments definitions
    
\makeatletter
\def\PY@reset{\let\PY@it=\relax \let\PY@bf=\relax%
    \let\PY@ul=\relax \let\PY@tc=\relax%
    \let\PY@bc=\relax \let\PY@ff=\relax}
\def\PY@tok#1{\csname PY@tok@#1\endcsname}
\def\PY@toks#1+{\ifx\relax#1\empty\else%
    \PY@tok{#1}\expandafter\PY@toks\fi}
\def\PY@do#1{\PY@bc{\PY@tc{\PY@ul{%
    \PY@it{\PY@bf{\PY@ff{#1}}}}}}}
\def\PY#1#2{\PY@reset\PY@toks#1+\relax+\PY@do{#2}}

\expandafter\def\csname PY@tok@gd\endcsname{\def\PY@tc##1{\textcolor[rgb]{0.63,0.00,0.00}{##1}}}
\expandafter\def\csname PY@tok@gu\endcsname{\let\PY@bf=\textbf\def\PY@tc##1{\textcolor[rgb]{0.50,0.00,0.50}{##1}}}
\expandafter\def\csname PY@tok@gt\endcsname{\def\PY@tc##1{\textcolor[rgb]{0.00,0.27,0.87}{##1}}}
\expandafter\def\csname PY@tok@gs\endcsname{\let\PY@bf=\textbf}
\expandafter\def\csname PY@tok@gr\endcsname{\def\PY@tc##1{\textcolor[rgb]{1.00,0.00,0.00}{##1}}}
\expandafter\def\csname PY@tok@cm\endcsname{\let\PY@it=\textit\def\PY@tc##1{\textcolor[rgb]{0.25,0.50,0.50}{##1}}}
\expandafter\def\csname PY@tok@vg\endcsname{\def\PY@tc##1{\textcolor[rgb]{0.10,0.09,0.49}{##1}}}
\expandafter\def\csname PY@tok@m\endcsname{\def\PY@tc##1{\textcolor[rgb]{0.40,0.40,0.40}{##1}}}
\expandafter\def\csname PY@tok@mh\endcsname{\def\PY@tc##1{\textcolor[rgb]{0.40,0.40,0.40}{##1}}}
\expandafter\def\csname PY@tok@go\endcsname{\def\PY@tc##1{\textcolor[rgb]{0.53,0.53,0.53}{##1}}}
\expandafter\def\csname PY@tok@ge\endcsname{\let\PY@it=\textit}
\expandafter\def\csname PY@tok@vc\endcsname{\def\PY@tc##1{\textcolor[rgb]{0.10,0.09,0.49}{##1}}}
\expandafter\def\csname PY@tok@il\endcsname{\def\PY@tc##1{\textcolor[rgb]{0.40,0.40,0.40}{##1}}}
\expandafter\def\csname PY@tok@cs\endcsname{\let\PY@it=\textit\def\PY@tc##1{\textcolor[rgb]{0.25,0.50,0.50}{##1}}}
\expandafter\def\csname PY@tok@cp\endcsname{\def\PY@tc##1{\textcolor[rgb]{0.74,0.48,0.00}{##1}}}
\expandafter\def\csname PY@tok@gi\endcsname{\def\PY@tc##1{\textcolor[rgb]{0.00,0.63,0.00}{##1}}}
\expandafter\def\csname PY@tok@gh\endcsname{\let\PY@bf=\textbf\def\PY@tc##1{\textcolor[rgb]{0.00,0.00,0.50}{##1}}}
\expandafter\def\csname PY@tok@ni\endcsname{\let\PY@bf=\textbf\def\PY@tc##1{\textcolor[rgb]{0.60,0.60,0.60}{##1}}}
\expandafter\def\csname PY@tok@nl\endcsname{\def\PY@tc##1{\textcolor[rgb]{0.63,0.63,0.00}{##1}}}
\expandafter\def\csname PY@tok@nn\endcsname{\let\PY@bf=\textbf\def\PY@tc##1{\textcolor[rgb]{0.00,0.00,1.00}{##1}}}
\expandafter\def\csname PY@tok@no\endcsname{\def\PY@tc##1{\textcolor[rgb]{0.53,0.00,0.00}{##1}}}
\expandafter\def\csname PY@tok@na\endcsname{\def\PY@tc##1{\textcolor[rgb]{0.49,0.56,0.16}{##1}}}
\expandafter\def\csname PY@tok@nb\endcsname{\def\PY@tc##1{\textcolor[rgb]{0.00,0.50,0.00}{##1}}}
\expandafter\def\csname PY@tok@nc\endcsname{\let\PY@bf=\textbf\def\PY@tc##1{\textcolor[rgb]{0.00,0.00,1.00}{##1}}}
\expandafter\def\csname PY@tok@nd\endcsname{\def\PY@tc##1{\textcolor[rgb]{0.67,0.13,1.00}{##1}}}
\expandafter\def\csname PY@tok@ne\endcsname{\let\PY@bf=\textbf\def\PY@tc##1{\textcolor[rgb]{0.82,0.25,0.23}{##1}}}
\expandafter\def\csname PY@tok@nf\endcsname{\def\PY@tc##1{\textcolor[rgb]{0.00,0.00,1.00}{##1}}}
\expandafter\def\csname PY@tok@si\endcsname{\let\PY@bf=\textbf\def\PY@tc##1{\textcolor[rgb]{0.73,0.40,0.53}{##1}}}
\expandafter\def\csname PY@tok@s2\endcsname{\def\PY@tc##1{\textcolor[rgb]{0.73,0.13,0.13}{##1}}}
\expandafter\def\csname PY@tok@vi\endcsname{\def\PY@tc##1{\textcolor[rgb]{0.10,0.09,0.49}{##1}}}
\expandafter\def\csname PY@tok@nt\endcsname{\let\PY@bf=\textbf\def\PY@tc##1{\textcolor[rgb]{0.00,0.50,0.00}{##1}}}
\expandafter\def\csname PY@tok@nv\endcsname{\def\PY@tc##1{\textcolor[rgb]{0.10,0.09,0.49}{##1}}}
\expandafter\def\csname PY@tok@s1\endcsname{\def\PY@tc##1{\textcolor[rgb]{0.73,0.13,0.13}{##1}}}
\expandafter\def\csname PY@tok@sh\endcsname{\def\PY@tc##1{\textcolor[rgb]{0.73,0.13,0.13}{##1}}}
\expandafter\def\csname PY@tok@sc\endcsname{\def\PY@tc##1{\textcolor[rgb]{0.73,0.13,0.13}{##1}}}
\expandafter\def\csname PY@tok@sx\endcsname{\def\PY@tc##1{\textcolor[rgb]{0.00,0.50,0.00}{##1}}}
\expandafter\def\csname PY@tok@bp\endcsname{\def\PY@tc##1{\textcolor[rgb]{0.00,0.50,0.00}{##1}}}
\expandafter\def\csname PY@tok@c1\endcsname{\let\PY@it=\textit\def\PY@tc##1{\textcolor[rgb]{0.25,0.50,0.50}{##1}}}
\expandafter\def\csname PY@tok@kc\endcsname{\let\PY@bf=\textbf\def\PY@tc##1{\textcolor[rgb]{0.00,0.50,0.00}{##1}}}
\expandafter\def\csname PY@tok@c\endcsname{\let\PY@it=\textit\def\PY@tc##1{\textcolor[rgb]{0.25,0.50,0.50}{##1}}}
\expandafter\def\csname PY@tok@mf\endcsname{\def\PY@tc##1{\textcolor[rgb]{0.40,0.40,0.40}{##1}}}
\expandafter\def\csname PY@tok@err\endcsname{\def\PY@bc##1{\setlength{\fboxsep}{0pt}\fcolorbox[rgb]{1.00,0.00,0.00}{1,1,1}{\strut ##1}}}
\expandafter\def\csname PY@tok@kd\endcsname{\let\PY@bf=\textbf\def\PY@tc##1{\textcolor[rgb]{0.00,0.50,0.00}{##1}}}
\expandafter\def\csname PY@tok@ss\endcsname{\def\PY@tc##1{\textcolor[rgb]{0.10,0.09,0.49}{##1}}}
\expandafter\def\csname PY@tok@sr\endcsname{\def\PY@tc##1{\textcolor[rgb]{0.73,0.40,0.53}{##1}}}
\expandafter\def\csname PY@tok@mo\endcsname{\def\PY@tc##1{\textcolor[rgb]{0.40,0.40,0.40}{##1}}}
\expandafter\def\csname PY@tok@kn\endcsname{\let\PY@bf=\textbf\def\PY@tc##1{\textcolor[rgb]{0.00,0.50,0.00}{##1}}}
\expandafter\def\csname PY@tok@mi\endcsname{\def\PY@tc##1{\textcolor[rgb]{0.40,0.40,0.40}{##1}}}
\expandafter\def\csname PY@tok@gp\endcsname{\let\PY@bf=\textbf\def\PY@tc##1{\textcolor[rgb]{0.00,0.00,0.50}{##1}}}
\expandafter\def\csname PY@tok@o\endcsname{\def\PY@tc##1{\textcolor[rgb]{0.40,0.40,0.40}{##1}}}
\expandafter\def\csname PY@tok@kr\endcsname{\let\PY@bf=\textbf\def\PY@tc##1{\textcolor[rgb]{0.00,0.50,0.00}{##1}}}
\expandafter\def\csname PY@tok@s\endcsname{\def\PY@tc##1{\textcolor[rgb]{0.73,0.13,0.13}{##1}}}
\expandafter\def\csname PY@tok@kp\endcsname{\def\PY@tc##1{\textcolor[rgb]{0.00,0.50,0.00}{##1}}}
\expandafter\def\csname PY@tok@w\endcsname{\def\PY@tc##1{\textcolor[rgb]{0.73,0.73,0.73}{##1}}}
\expandafter\def\csname PY@tok@kt\endcsname{\def\PY@tc##1{\textcolor[rgb]{0.69,0.00,0.25}{##1}}}
\expandafter\def\csname PY@tok@ow\endcsname{\let\PY@bf=\textbf\def\PY@tc##1{\textcolor[rgb]{0.67,0.13,1.00}{##1}}}
\expandafter\def\csname PY@tok@sb\endcsname{\def\PY@tc##1{\textcolor[rgb]{0.73,0.13,0.13}{##1}}}
\expandafter\def\csname PY@tok@k\endcsname{\let\PY@bf=\textbf\def\PY@tc##1{\textcolor[rgb]{0.00,0.50,0.00}{##1}}}
\expandafter\def\csname PY@tok@se\endcsname{\let\PY@bf=\textbf\def\PY@tc##1{\textcolor[rgb]{0.73,0.40,0.13}{##1}}}
\expandafter\def\csname PY@tok@sd\endcsname{\let\PY@it=\textit\def\PY@tc##1{\textcolor[rgb]{0.73,0.13,0.13}{##1}}}

\def\PYZbs{\char`\\}
\def\PYZus{\char`\_}
\def\PYZob{\char`\{}
\def\PYZcb{\char`\}}
\def\PYZca{\char`\^}
\def\PYZam{\char`\&}
\def\PYZlt{\char`\<}
\def\PYZgt{\char`\>}
\def\PYZsh{\char`\#}
\def\PYZpc{\char`\%}
\def\PYZdl{\char`\$}
\def\PYZhy{\char`\-}
\def\PYZsq{\char`\'}
\def\PYZdq{\char`\"}
\def\PYZti{\char`\~}
% for compatibility with earlier versions
\def\PYZat{@}
\def\PYZlb{[}
\def\PYZrb{]}
\makeatother


    %Set pygments styles if needed...
    
        \definecolor{nbframe-border}{rgb}{0.867,0.867,0.867}
        \definecolor{nbframe-bg}{rgb}{0.969,0.969,0.969}
        \definecolor{nbframe-in-prompt}{rgb}{0.0,0.0,0.502}
        \definecolor{nbframe-out-prompt}{rgb}{0.545,0.0,0.0}

        \newenvironment{ColorVerbatim}
        {\begin{mdframed}[%
            roundcorner=1.0pt, %
            backgroundcolor=nbframe-bg, %
            userdefinedwidth=1\linewidth, %
            leftmargin=0.1\linewidth, %
            innerleftmargin=0pt, %
            innerrightmargin=0pt, %
            linecolor=nbframe-border, %
            linewidth=1pt, %
            usetwoside=false, %
            everyline=true, %
            innerlinewidth=3pt, %
            innerlinecolor=nbframe-bg, %
            middlelinewidth=1pt, %
            middlelinecolor=nbframe-bg, %
            outerlinewidth=0.5pt, %
            outerlinecolor=nbframe-border, %
            needspace=0pt
        ]}
        {\end{mdframed}}
        
        \newenvironment{InvisibleVerbatim}
        {\begin{mdframed}[leftmargin=0.1\linewidth,innerleftmargin=3pt,innerrightmargin=3pt, userdefinedwidth=1\linewidth, linewidth=0pt, linecolor=white, usetwoside=false]}
        {\end{mdframed}}

        \renewenvironment{Verbatim}[1][\unskip]
        {\begin{alltt}\smaller}
        {\end{alltt}}
    

    % Help prevent overflowing lines due to urls and other hard-to-break 
    % entities.  This doesn't catch everything...
    \sloppy

    % Document level variables
    \title{Compressive Sampling}
    \date{April 27, 2013}
    \release{}
    \author{}
    \renewcommand{\releasename}{}

    % TODO: Add option for the user to specify a logo for his/her export.
    \newcommand{\sphinxlogo}{}

    % Make the index page of the document.
    \makeindex

    % Import sphinx document type specifics.
     


% Body

    % Start of the document
    \begin{document}

        
            \maketitle
        

        


        \section{Compressive sampling Overview
}In our previous discussion, we saw that imposing bandlimited-ness on our
class of signals permits point-wise sampling of our signal and then
later perfect reconstruction. It turns out that by imposing
\emph{sparsity} we can also obtain perfect reconstruction irrespective
of whether or not we have satsified the sampling rate limits imposed by
Shannon's sampling theorem. This has extremely important in practice
because many signals are naturally sparse so that collecting samples at
high rates only to dump most of them as the signal is compressed is
expensive and wasteful.
\section{What Are Sparse Signals?
}Let's carefully discuss what we mean by \emph{sparse} in this context. A
signal $f$ is sparse if it can be expressed in very few nonzero
components ($\mathbf{s}$) with respect to a given basis ($\mathbf{\Psi}$
). In other words, in matrix-vector language:

$\mathbf{f} = \mathbf{\Psi} \mathbf{s}$

where $|| \mathbf{s} ||_0 \leq N$ where $N$ is the length of the vector
and $|| \cdot||_0$ counts the number of nonzero elements in
$\mathbf{s}$. Furthermore, we don't actually collect $N$ samples
point-wise as we did in the Shannon sampling case. Rather, we measure
$\mathbf{f}$ indirectly as $\mathbf{y}$ with another matrix as in:

$\mathbf{y} = \mathbf{\Phi f} = \mathbf{\Phi} \mathbf{\Psi} \mathbf{s} = \mathbf{\Theta s}$

where $\mathbf{\Theta}$ is an $M \times N$ matrix and $M < N$ is the
number of measurements. This setup means we have two problems to solve.
First, how to design a \emph{stable} measurement matrix $\mathbf{\Phi}$
and then, second, how to reconstruct $\mathbf{f}$ from $\mathbf{y}$.

This may look like a standard linear algebra problem but since
$\mathbf{\Theta}$ has fewer rows than columns, the solution is
necessarily ill-posed. This is where we inject the sparsity concept!
Suppose that $f$ is $K$-sparse ( $||f||_0=K$ ), then if we somehow knew
\emph{which} $K$ columns of $\mathbf{\Theta}$ matched the $K$ non-zero
entries in $\mathbf{s}$, then $\mathbf{\Theta}$ would be $M \times K$
where we could make $M > K$ and then have a stable inverse.

This bit of reasoning is encapsulated in the following statement for any
vector $\mathbf{v}$ sharing the same $K$ non-zero entries as
$\mathbf{s}$, we have

\[1-\epsilon \leq \frac{||  \mathbf{\Theta v} ||_2}{|| \mathbf{v}  ||_2} \leq 1+\epsilon \]

which is another way of saying that $\mathbf{\Theta}$ preserves the
lengths of $K$-sparse vectors. Of course we don't know ahead of time
which $K$ components to use, but it turns out that this condition is
sufficient for a stable inverse of $\mathbf{\Theta}$ if it holds for any
$3K$-sparse vector $\mathbf{v}$. This is the \emph{Restricted Isometry
Property} (RIP). Unfortunately, in order to use this sufficient
condition, we would have to propose a $\mathbf{\Theta}$ and then check
all possible combinations of nonzero entries in the $N$-length vector
$\mathbf{v}$. As you may guess, this is prohibitive.

Alternatively, we can approach stability by defining \emph{incoherence}
between the measurement matrix $\mathbf{\Phi}$ and the sparse basis
$\mathbf{\Psi}$ as when any of the columns of one cannot be expressed as
a small subset of the columns of the other. For example, if we have
delta-spikes for $\mathbf{\Phi}$ as the row-truncated identity matrix

\[\mathbf{\Phi} = \mathbf{I}_{M \times N} \]

and the discrete Fourier transform matrix for $\mathbf{\Psi}$ as

$\mathbf{\Psi} = \begin{bmatrix}\\\\ e^{-j 2\pi k n/N}\\\\ \end{bmatrix}_{N \times N}$

Then we could not write any of the columns of $\mathbf{\Phi}$ using just
a few of the columns of $\mathbf{\Psi}$.

It turns out that picking the measuring $M \times N$ matrix randomly
according to a Gaussian zero-mean, $1/N$ variance distribution and using
the identity matrix as $\mathbf{\Phi}$, that the resulting
$\mathbf{\Theta}$ matrix can be shown to satisfy RIP with a high
probability. This means that we can recover $N$-length $K$-sparse
signals with a high probability from just $M \ge c K \log (N/K)$ samples
where $c$ is a small constant. Furthermore, it also turns out that we
can use any orthonormal basis for $\mathbf{\Phi}$, not just the identity
matrix, and these relations will all still hold.
\section{Reconstructing Sparse Signals
}Now that we have a way, by using random matrices, to satisfy the RIP, we
are ready to consider the reconstruction problem. The first impulse is
to compute the least-squares solution to this problem as

\[ \mathbf{s}^* = \mathbf{\Theta}^T (\mathbf{\Theta}\mathbf{\Theta}^T)^{-1}\mathbf{y} \]

But a moment's thought may convince you that since $\mathbf{\Theta}$ is
a random matrix, most likely with lots of non-zero entries, it is highly
unlikely that $\mathbf{s}^*$ will turn out to be sparse. There is
actually a deeper geometric intuition as to why this happens, but let's
first consider another way of solving this so that the $\mathbf{s}^*$ is
$K$-sparse. Suppose instead we shuffle through combinations of $K$
nonzero entries in $\mathbf{s}$ until we satisfy the measurements
$\mathbf{y}$. Stated mathematically, this means

\[ \mathbf{s}^* = argmin || \mathbf{s}^* ||_0  \]

where

\[ \mathbf{\Theta} \mathbf{s}^* = \mathbf{y} \]

It can be shown that with $M=K+1$ iid Gaussian measurements, this
optimization will recover a $K$-sparse signal exactly with high
probability. Unfortunately, this is numerically unstable in addition to
being an NP-complete problem.

Thus, we need another tractable way to approach this problem. It turns
out that when a signal is sparse, it usually means that the nonzero
terms are highly asymmetric meaning that if there are $K$ terms, then
most likely there is one term that is dominant (i.e.~of much larger
magnitude) and that dwarfs the other nonzero terms. Geometrically, this
means that in $N$-dimensional space, the sparse signal is very close to
one (or, maybe just a few) of the axes.

It turns out that one can bypass this combinatorial problem using $L_1$
minimization. To examine this, let's digress and look at the main
difference between $L_2$ and $L_1$ minimization problems.
reference: \texttt{http://users.ece.gatech.edu/justin/ssp2007}
\section{$L\_2$ vs. $L\_1$ Optimization
}The classic constrained least squares problem is the following:

min $||\mathbf{x}||_2^2$

where $x_1 + 2 x_2 = 1$

with corresponding solution illustrated below.


    % Make sure that atleast 4 lines are below the HR
    \needspace{4\baselineskip}

    
        \vspace{6pt}
        \makebox[0.1\linewidth]{\smaller\hfill\tt\color{nbframe-in-prompt}In\hspace{4pt}{[}1{]}:\hspace{4pt}}\\*
        \vspace{-2.65\baselineskip}
        \begin{ColorVerbatim}
            \vspace{-0.7\baselineskip}
            \begin{Verbatim}[commandchars=\\\{\}]
\PY{k+kn}{from} \PY{n+nn}{\PYZus{}\PYZus{}future\PYZus{}\PYZus{}} \PY{k+kn}{import} \PY{n}{division}
\PY{k+kn}{from} \PY{n+nn}{matplotlib.patches} \PY{k+kn}{import} \PY{n}{Circle}
\PY{n}{x1} \PY{o}{=} \PY{n}{linspace}\PY{p}{(}\PY{o}{\PYZhy{}}\PY{l+m+mi}{1}\PY{p}{,}\PY{l+m+mi}{1}\PY{p}{,}\PY{l+m+mi}{10}\PY{p}{)}
\PY{n}{fig}\PY{o}{=}\PY{n}{figure}\PY{p}{(}\PY{p}{)}
\PY{n}{ax}\PY{o}{=}\PY{n}{fig}\PY{o}{.}\PY{n}{add\PYZus{}subplot}\PY{p}{(}\PY{l+m+mi}{111}\PY{p}{)}
\PY{n}{ax}\PY{o}{.}\PY{n}{plot}\PY{p}{(}\PY{n}{x1}\PY{p}{,}\PY{p}{(}\PY{l+m+mi}{1}\PY{o}{\PYZhy{}}\PY{n}{x1}\PY{p}{)}\PY{o}{/}\PY{l+m+mi}{2}\PY{p}{)}
\PY{n}{ax}\PY{o}{.}\PY{n}{add\PYZus{}patch}\PY{p}{(}\PY{n}{Circle}\PY{p}{(}\PY{p}{(}\PY{l+m+mi}{0}\PY{p}{,}\PY{l+m+mi}{0}\PY{p}{)}\PY{p}{,}\PY{l+m+mi}{1}\PY{o}{/}\PY{n}{sqrt}\PY{p}{(}\PY{l+m+mi}{5}\PY{p}{)}\PY{p}{,}\PY{n}{alpha}\PY{o}{=}\PY{l+m+mf}{0.3}\PY{p}{)}\PY{p}{)}
\PY{n}{ax}\PY{o}{.}\PY{n}{plot}\PY{p}{(}\PY{l+m+mi}{1}\PY{o}{/}\PY{l+m+mi}{5}\PY{p}{,}\PY{l+m+mi}{2}\PY{o}{/}\PY{l+m+mi}{5}\PY{p}{,}\PY{l+s}{\PYZsq{}}\PY{l+s}{rs}\PY{l+s}{\PYZsq{}}\PY{p}{)}
\PY{n}{ax}\PY{o}{.}\PY{n}{axis}\PY{p}{(}\PY{l+s}{\PYZsq{}}\PY{l+s}{equal}\PY{l+s}{\PYZsq{}}\PY{p}{)}
\PY{n}{ax}\PY{o}{.}\PY{n}{set\PYZus{}xlabel}\PY{p}{(}\PY{l+s}{\PYZsq{}}\PY{l+s}{\PYZdl{}x\PYZus{}1\PYZdl{}}\PY{l+s}{\PYZsq{}}\PY{p}{,}\PY{n}{fontsize}\PY{o}{=}\PY{l+m+mi}{24}\PY{p}{)}
\PY{n}{ax}\PY{o}{.}\PY{n}{set\PYZus{}ylabel}\PY{p}{(}\PY{l+s}{\PYZsq{}}\PY{l+s}{\PYZdl{}x\PYZus{}2\PYZdl{}}\PY{l+s}{\PYZsq{}}\PY{p}{,}\PY{n}{fontsize}\PY{o}{=}\PY{l+m+mi}{24}\PY{p}{)}
\PY{n}{ax}\PY{o}{.}\PY{n}{grid}\PY{p}{(}\PY{p}{)}
\end{Verbatim}

            
                \vspace{-0.2\baselineskip}
            
        \end{ColorVerbatim}
    

    

        % If the first block is an image, minipage the image.  Else
        % request a certain amount of space for the input text.
        \needspace{4\baselineskip}
        
        

            % Add document contents.
            
                \begin{InvisibleVerbatim}
                \vspace{-0.5\baselineskip}\begin{center}
    \includegraphics[max size={\textwidth}{\textheight}]{_fig_900.png}
    \par
    \end{center}
            \end{InvisibleVerbatim}
            
        
    
Note that the line is the constraint so that any solution to this
problem must be on this line (i.e.~satisfy the constraint). The $L_2$
solution is the one that just touches the perimeter of the circle. This
is because, in $L_2$, the unit-ball has the shape of a circle and
represents all solutions of a fixed $L_2$ length. Thus, the one of
smallest length that intersects the line is the one that satisfies the
stated minimization problem. Intuitively, this means that we
\emph{inflate} a ball at the origin and stop when it touches the
contraint. The point of contact is our $L_2$ minimization solution.

Now, let's do same problem in $L_1$ norm

min $||\mathbf{x}||_1=|x_1|+|x_2|$

where $x_1 + 2 x_2 = 1$

In this case the constant-norm unit-ball contour in the $L_1$ norm is a
diamond-shape instead of a circle. Comparing the graph below to the last
shows that the solutions found are different. Geometrically, this is
because the line tilts over in such a way that the inflating circular
$L_2$ ball hits a point of tangency that is different from the $L_1$
ball because the $L_1$ ball creeps out mainly along the principal axes
and is less influenced by the tilt of the line. This effect is much more
pronounced in higher $N$-dimensional spaces where $L_1$-balls get more
\emph{spikey}.

The fact that the $L_1$ problem is less sensitive to the tilt of the
line is crucial since that tilt (i.e.~orientation) is random due the
choice of random measurement matrices. So, for this problem to be
well-posed, we need to \emph{not} be influenced by the orientation of
any particular choice of random matrix and this is what casting this as
a $L_1$ minimization provides.


    % Make sure that atleast 4 lines are below the HR
    \needspace{4\baselineskip}

    
        \vspace{6pt}
        \makebox[0.1\linewidth]{\smaller\hfill\tt\color{nbframe-in-prompt}In\hspace{4pt}{[}2{]}:\hspace{4pt}}\\*
        \vspace{-2.65\baselineskip}
        \begin{ColorVerbatim}
            \vspace{-0.7\baselineskip}
            \begin{Verbatim}[commandchars=\\\{\}]
\PY{k+kn}{from} \PY{n+nn}{matplotlib.patches} \PY{k+kn}{import} \PY{n}{Rectangle}
\PY{k+kn}{import} \PY{n+nn}{matplotlib.patches}
\PY{k+kn}{import} \PY{n+nn}{matplotlib.transforms}

\PY{n}{r}\PY{o}{=}\PY{n}{matplotlib}\PY{o}{.}\PY{n}{patches}\PY{o}{.}\PY{n}{RegularPolygon}\PY{p}{(}\PY{p}{(}\PY{l+m+mi}{0}\PY{p}{,}\PY{l+m+mi}{0}\PY{p}{)}\PY{p}{,}\PY{l+m+mi}{4}\PY{p}{,}\PY{l+m+mi}{1}\PY{o}{/}\PY{l+m+mi}{2}\PY{p}{,}\PY{n}{pi}\PY{o}{/}\PY{l+m+mi}{2}\PY{p}{,}\PY{n}{alpha}\PY{o}{=}\PY{l+m+mf}{0.5}\PY{p}{)}

\PY{n}{fig}\PY{o}{=}\PY{n}{figure}\PY{p}{(}\PY{p}{)}
\PY{n}{ax}\PY{o}{=}\PY{n}{fig}\PY{o}{.}\PY{n}{add\PYZus{}subplot}\PY{p}{(}\PY{l+m+mi}{111}\PY{p}{)}
\PY{n}{ax}\PY{o}{.}\PY{n}{plot}\PY{p}{(}\PY{n}{x1}\PY{p}{,}\PY{p}{(}\PY{l+m+mi}{1}\PY{o}{\PYZhy{}}\PY{n}{x1}\PY{p}{)}\PY{o}{/}\PY{l+m+mi}{2}\PY{p}{)}
\PY{n}{ax}\PY{o}{.}\PY{n}{plot}\PY{p}{(}\PY{l+m+mi}{0}\PY{p}{,}\PY{l+m+mi}{1}\PY{o}{/}\PY{l+m+mi}{2}\PY{p}{,}\PY{l+s}{\PYZsq{}}\PY{l+s}{rs}\PY{l+s}{\PYZsq{}}\PY{p}{)}
\PY{n}{ax}\PY{o}{.}\PY{n}{add\PYZus{}patch}\PY{p}{(}\PY{n}{r}\PY{p}{)}
\PY{n}{ax}\PY{o}{.}\PY{n}{grid}\PY{p}{(}\PY{p}{)}
\PY{n}{ax}\PY{o}{.}\PY{n}{set\PYZus{}xlabel}\PY{p}{(}\PY{l+s}{\PYZsq{}}\PY{l+s}{\PYZdl{}x\PYZus{}1\PYZdl{}}\PY{l+s}{\PYZsq{}}\PY{p}{,}\PY{n}{fontsize}\PY{o}{=}\PY{l+m+mi}{24}\PY{p}{)}
\PY{n}{ax}\PY{o}{.}\PY{n}{set\PYZus{}ylabel}\PY{p}{(}\PY{l+s}{\PYZsq{}}\PY{l+s}{\PYZdl{}x\PYZus{}2\PYZdl{}}\PY{l+s}{\PYZsq{}}\PY{p}{,}\PY{n}{fontsize}\PY{o}{=}\PY{l+m+mi}{24}\PY{p}{)}
\PY{n}{ax}\PY{o}{.}\PY{n}{axis}\PY{p}{(}\PY{l+s}{\PYZsq{}}\PY{l+s}{equal}\PY{l+s}{\PYZsq{}}\PY{p}{)}
\end{Verbatim}

            
                \vspace{-0.2\baselineskip}
            
        \end{ColorVerbatim}
    

    

        % If the first block is an image, minipage the image.  Else
        % request a certain amount of space for the input text.
        \needspace{4\baselineskip}
        
        

            % Add document contents.
            
                \makebox[0.1\linewidth]{\smaller\hfill\tt\color{nbframe-out-prompt}Out\hspace{4pt}{[}2{]}:\hspace{4pt}}\\*
                \vspace{-2.55\baselineskip}\begin{InvisibleVerbatim}
                \vspace{-0.5\baselineskip}
    \begin{alltt}(-1.0, 1.0, -0.60000000000000009, 1.2)\end{alltt}

            \end{InvisibleVerbatim}
            
                \begin{InvisibleVerbatim}
                \vspace{-0.5\baselineskip}\begin{center}
    \includegraphics[max size={\textwidth}{\textheight}]{_fig_1100.png}
    \par
    \end{center}
            \end{InvisibleVerbatim}
            
        
    
To explore this a bit, let's consider using the \texttt{cvxopt} package
(Python ver 2.6 used here). This can be cast as a linear programming
problem as follows:

min $||\mathbf{t}||_1 = |t_1| + |t_2|$

subject to:

$-t_1 < x_1 < t_1$

$-t_2 < x_2 < t_2$

$x_1 + 2 x_2 = 1$

$t_1 > 0$

$t_2 > 0$

where the last two constraints are already implied by the first two and
are written out just for clarity. This can be implemented and solved in
\texttt{cvxopt} as the following:


    % Make sure that atleast 4 lines are below the HR
    \needspace{4\baselineskip}

    
        \vspace{6pt}
        \makebox[0.1\linewidth]{\smaller\hfill\tt\color{nbframe-in-prompt}In\hspace{4pt}{[}3{]}:\hspace{4pt}}\\*
        \vspace{-2.65\baselineskip}
        \begin{ColorVerbatim}
            \vspace{-0.7\baselineskip}
            \begin{Verbatim}[commandchars=\\\{\}]
\PY{k+kn}{from} \PY{n+nn}{cvxopt} \PY{k+kn}{import} \PY{n}{matrix} \PY{k}{as} \PY{n}{matrx} \PY{c}{\PYZsh{} don\PYZsq{}t overrite numpy matrix class}
\PY{k+kn}{from} \PY{n+nn}{cvxopt} \PY{k+kn}{import} \PY{n}{solvers}

\PY{c}{\PYZsh{}t1,x1,t2,x2}
\PY{n}{c} \PY{o}{=} \PY{n}{matrx}\PY{p}{(}\PY{p}{[}\PY{l+m+mi}{1}\PY{p}{,}\PY{l+m+mi}{0}\PY{p}{,}\PY{l+m+mi}{1}\PY{p}{,}\PY{l+m+mi}{0}\PY{p}{]}\PY{p}{,}\PY{p}{(}\PY{l+m+mi}{4}\PY{p}{,}\PY{l+m+mi}{1}\PY{p}{)}\PY{p}{,}\PY{l+s}{\PYZsq{}}\PY{l+s}{d}\PY{l+s}{\PYZsq{}}\PY{p}{)} 
\PY{n}{G} \PY{o}{=} \PY{n}{matrx}\PY{p}{(}\PY{p}{[}   \PY{p}{[}\PY{o}{\PYZhy{}}\PY{l+m+mi}{1}\PY{p}{,}  \PY{o}{\PYZhy{}}\PY{l+m+mi}{1}\PY{p}{,} \PY{l+m+mi}{0}\PY{p}{,}  \PY{l+m+mi}{0}\PY{p}{]}\PY{p}{,}  \PY{c}{\PYZsh{}column\PYZhy{}0}
              \PY{p}{[} \PY{l+m+mi}{1}\PY{p}{,}  \PY{o}{\PYZhy{}}\PY{l+m+mi}{1}\PY{p}{,} \PY{l+m+mi}{0}\PY{p}{,}  \PY{l+m+mi}{0}\PY{p}{]}\PY{p}{,}  \PY{c}{\PYZsh{}column\PYZhy{}1}
              \PY{p}{[} \PY{l+m+mi}{0}\PY{p}{,}   \PY{l+m+mi}{0}\PY{p}{,} \PY{o}{\PYZhy{}}\PY{l+m+mi}{1}\PY{p}{,}\PY{o}{\PYZhy{}}\PY{l+m+mi}{1}\PY{p}{]}\PY{p}{,}  \PY{c}{\PYZsh{}column\PYZhy{}2}
              \PY{p}{[} \PY{l+m+mi}{0}\PY{p}{,}   \PY{l+m+mi}{0}\PY{p}{,}  \PY{l+m+mi}{1}\PY{p}{,}\PY{o}{\PYZhy{}}\PY{l+m+mi}{1}\PY{p}{]}\PY{p}{,}  \PY{c}{\PYZsh{}column\PYZhy{}3}
           \PY{p}{]}\PY{p}{,}\PY{p}{(}\PY{l+m+mi}{4}\PY{p}{,}\PY{l+m+mi}{4}\PY{p}{)}\PY{p}{,}\PY{l+s}{\PYZsq{}}\PY{l+s}{d}\PY{l+s}{\PYZsq{}}\PY{p}{)}

\PY{n}{h} \PY{o}{=} \PY{n}{matrx}\PY{p}{(}\PY{p}{[}\PY{l+m+mi}{0}\PY{p}{,}\PY{l+m+mi}{0}\PY{p}{,}\PY{l+m+mi}{0}\PY{p}{,}\PY{l+m+mi}{0}\PY{p}{]}\PY{p}{,}\PY{p}{(}\PY{l+m+mi}{4}\PY{p}{,}\PY{l+m+mi}{1}\PY{p}{)}\PY{p}{,}\PY{l+s}{\PYZsq{}}\PY{l+s}{d}\PY{l+s}{\PYZsq{}}\PY{p}{)} \PY{c}{\PYZsh{} (4,1) is 4\PYZhy{}rows,1\PYZhy{}column, \PYZsq{}d\PYZsq{} is float type spec}
\PY{n}{A} \PY{o}{=} \PY{n}{matrx}\PY{p}{(}\PY{p}{[}\PY{l+m+mi}{0}\PY{p}{,}\PY{l+m+mi}{1}\PY{p}{,}\PY{l+m+mi}{0}\PY{p}{,}\PY{l+m+mi}{2}\PY{p}{]}\PY{p}{,}\PY{p}{(}\PY{l+m+mi}{1}\PY{p}{,}\PY{l+m+mi}{4}\PY{p}{)}\PY{p}{,}\PY{l+s}{\PYZsq{}}\PY{l+s}{d}\PY{l+s}{\PYZsq{}}\PY{p}{)}
\PY{n}{b} \PY{o}{=} \PY{n}{matrx}\PY{p}{(}\PY{p}{[}\PY{l+m+mi}{1}\PY{p}{]}\PY{p}{,}\PY{p}{(}\PY{l+m+mi}{1}\PY{p}{,}\PY{l+m+mi}{1}\PY{p}{)}\PY{p}{,}\PY{l+s}{\PYZsq{}}\PY{l+s}{d}\PY{l+s}{\PYZsq{}}\PY{p}{)}
\PY{n}{sol} \PY{o}{=} \PY{n}{solvers}\PY{o}{.}\PY{n}{lp}\PY{p}{(}\PY{n}{c}\PY{p}{,} \PY{n}{G}\PY{p}{,} \PY{n}{h}\PY{p}{,}\PY{n}{A}\PY{p}{,}\PY{n}{b}\PY{p}{)}
\PY{n}{x1}\PY{o}{=}\PY{n}{sol}\PY{p}{[}\PY{l+s}{\PYZsq{}}\PY{l+s}{x}\PY{l+s}{\PYZsq{}}\PY{p}{]}\PY{p}{[}\PY{l+m+mi}{1}\PY{p}{]}
\PY{n}{x2}\PY{o}{=}\PY{n}{sol}\PY{p}{[}\PY{l+s}{\PYZsq{}}\PY{l+s}{x}\PY{l+s}{\PYZsq{}}\PY{p}{]}\PY{p}{[}\PY{l+m+mi}{3}\PY{p}{]}
\PY{k}{print} \PY{l+s}{\PYZsq{}}\PY{l+s}{x=}\PY{l+s+si}{\PYZpc{}3.2f}\PY{l+s}{\PYZsq{}}\PY{o}{\PYZpc{}} \PY{n}{x1}
\PY{k}{print} \PY{l+s}{\PYZsq{}}\PY{l+s}{y=}\PY{l+s+si}{\PYZpc{}3.2f}\PY{l+s}{\PYZsq{}}\PY{o}{\PYZpc{}} \PY{n}{x2}
\end{Verbatim}

            
                \vspace{-0.2\baselineskip}
            
        \end{ColorVerbatim}
    

    

        % If the first block is an image, minipage the image.  Else
        % request a certain amount of space for the input text.
        \needspace{4\baselineskip}
        
        

            % Add document contents.
            
                \begin{InvisibleVerbatim}
                \vspace{-0.5\baselineskip}
    \begin{alltt}     pcost       dcost       gap    pres   dres   k/t
 0:  0.0000e+00 -0.0000e+00  3e+00  3e+00  1e-16  1e+00
 1:  2.3609e-01  2.3386e-01  5e-01  5e-01  1e-16  2e-01
 2:  4.9833e-01  4.9734e-01  5e-02  4e-02  5e-15  1e-02
 3:  4.9998e-01  4.9997e-01  5e-04  5e-04  2e-15  2e-04
 4:  5.0000e-01  5.0000e-01  5e-06  5e-06  6e-16  2e-06
 5:  5.0000e-01  5.0000e-01  5e-08  5e-08  9e-16  2e-08
Optimal solution found.
x=0.00
y=0.50
\end{alltt}

            \end{InvisibleVerbatim}
            
        
    
\section{Example Gaussian Random matrices
}Let's try out our earlier result about random Gaussian matrices and see
if we can reconstruct an unknown $\mathbf{s}$ vector using $L_1$
minimization.


    % Make sure that atleast 4 lines are below the HR
    \needspace{4\baselineskip}

    
        \vspace{6pt}
        \makebox[0.1\linewidth]{\smaller\hfill\tt\color{nbframe-in-prompt}In\hspace{4pt}{[}56{]}:\hspace{4pt}}\\*
        \vspace{-2.65\baselineskip}
        \begin{ColorVerbatim}
            \vspace{-0.7\baselineskip}
            \begin{Verbatim}[commandchars=\\\{\}]
\PY{k+kn}{import} \PY{n+nn}{scipy.linalg}

\PY{k}{def} \PY{n+nf}{rearrange\PYZus{}G}\PY{p}{(} \PY{n}{x} \PY{p}{)}\PY{p}{:} 
    \PY{l+s}{\PYZsq{}}\PY{l+s}{setup to put inequalities matrix with last 1/2 of elements as main variables}\PY{l+s}{\PYZsq{}}
    \PY{n}{n}\PY{o}{=}\PY{n}{x}\PY{o}{.}\PY{n}{shape}\PY{p}{[}\PY{l+m+mi}{0}\PY{p}{]}
    \PY{k}{return} \PY{n}{hstack}\PY{p}{(}\PY{p}{[}\PY{n}{x}\PY{p}{[}\PY{p}{:}\PY{p}{,}\PY{n}{arange}\PY{p}{(}\PY{l+m+mi}{0}\PY{p}{,}\PY{n}{n}\PY{p}{,}\PY{l+m+mi}{2}\PY{p}{)}\PY{o}{+}\PY{l+m+mi}{1}\PY{p}{]}\PY{p}{,} \PY{n}{x}\PY{p}{[}\PY{p}{:}\PY{p}{,}\PY{n}{arange}\PY{p}{(}\PY{l+m+mi}{0}\PY{p}{,}\PY{n}{n}\PY{p}{,}\PY{l+m+mi}{2}\PY{p}{)}\PY{p}{]}\PY{p}{]}\PY{p}{)}

\PY{n}{K}\PY{o}{=}\PY{l+m+mi}{2} \PY{c}{\PYZsh{} components}
\PY{n}{Nf}\PY{o}{=}\PY{l+m+mi}{128} \PY{c}{\PYZsh{} number of samples}
\PY{n}{M} \PY{o}{=} \PY{l+m+mi}{12} \PY{c}{\PYZsh{} \PYZgt{} K log2(Nf/K); num of measurements}
\PY{n}{s}\PY{o}{=}\PY{n}{zeros}\PY{p}{(}\PY{p}{(}\PY{n}{Nf}\PY{p}{,}\PY{l+m+mi}{1}\PY{p}{)}\PY{p}{)} \PY{c}{\PYZsh{} sparse vector we want to find}
\PY{n}{s}\PY{p}{[}\PY{l+m+mi}{0}\PY{p}{]}\PY{o}{=}\PY{l+m+mi}{1} \PY{c}{\PYZsh{} set the K nonzero entries}
\PY{n}{s}\PY{p}{[}\PY{l+m+mi}{1}\PY{p}{]}\PY{o}{=}\PY{l+m+mf}{0.5}
\PY{n}{np}\PY{o}{.}\PY{n}{random}\PY{o}{.}\PY{n}{seed}\PY{p}{(}\PY{l+m+mi}{5489}\PY{p}{)} \PY{c}{\PYZsh{} set random seed for reproducibility}
\PY{n}{Phi} \PY{o}{=} \PY{n}{matrix}\PY{p}{(}\PY{n}{randn}\PY{p}{(}\PY{n}{M}\PY{p}{,}\PY{n}{Nf}\PY{p}{)}\PY{o}{*}\PY{n}{sqrt}\PY{p}{(}\PY{l+m+mi}{1}\PY{o}{/}\PY{n}{Nf}\PY{p}{)}\PY{p}{)} \PY{c}{\PYZsh{} random Gaussian matrix}
\PY{n}{y}\PY{o}{=}\PY{n}{Phi}\PY{o}{*}\PY{n}{s} \PY{c}{\PYZsh{} measurements}

\PY{c}{\PYZsh{}\PYZhy{}\PYZhy{} setup L1 minimization problem \PYZhy{}\PYZhy{} }

\PY{c}{\PYZsh{} inequalities matrix with }
\PY{n}{G}\PY{o}{=}\PY{n}{matrx}\PY{p}{(}\PY{n}{rearrange\PYZus{}G}\PY{p}{(}\PY{n}{scipy}\PY{o}{.}\PY{n}{linalg}\PY{o}{.}\PY{n}{block\PYZus{}diag}\PY{p}{(}\PY{o}{*}\PY{p}{[}\PY{n}{matrix}\PY{p}{(}\PY{p}{[}\PY{p}{[}\PY{o}{\PYZhy{}}\PY{l+m+mi}{1}\PY{p}{,}\PY{o}{\PYZhy{}}\PY{l+m+mi}{1}\PY{p}{]}\PY{p}{,}\PY{p}{[}\PY{l+m+mi}{1}\PY{p}{,}\PY{o}{\PYZhy{}}\PY{l+m+mf}{1.0}\PY{p}{]}\PY{p}{]}\PY{p}{)}\PY{p}{,}\PY{p}{]}\PY{o}{*}\PY{n}{Nf}\PY{p}{)} \PY{p}{)}\PY{p}{)}
\PY{c}{\PYZsh{} objective function row\PYZhy{}matrix}
\PY{n}{c}\PY{o}{=}\PY{n}{matrx}\PY{p}{(}\PY{n}{hstack}\PY{p}{(}\PY{p}{[}\PY{n}{ones}\PY{p}{(}\PY{n}{Nf}\PY{p}{)}\PY{p}{,}\PY{n}{zeros}\PY{p}{(}\PY{n}{Nf}\PY{p}{)}\PY{p}{]}\PY{p}{)}\PY{p}{)}
\PY{c}{\PYZsh{} RHS for inequalities}
\PY{n}{h} \PY{o}{=} \PY{n}{matrx}\PY{p}{(}\PY{p}{[}\PY{l+m+mf}{0.0}\PY{p}{,}\PY{p}{]}\PY{o}{*}\PY{p}{(}\PY{n}{Nf}\PY{o}{*}\PY{l+m+mi}{2}\PY{p}{)}\PY{p}{,}\PY{p}{(}\PY{n}{Nf}\PY{o}{*}\PY{l+m+mi}{2}\PY{p}{,}\PY{l+m+mi}{1}\PY{p}{)}\PY{p}{,}\PY{l+s}{\PYZsq{}}\PY{l+s}{d}\PY{l+s}{\PYZsq{}}\PY{p}{)} 
\PY{c}{\PYZsh{} equality constraint matrix}
\PY{n}{A} \PY{o}{=} \PY{n}{matrx}\PY{p}{(}\PY{n}{hstack}\PY{p}{(}\PY{p}{[}\PY{n}{Phi}\PY{o}{*}\PY{l+m+mi}{0}\PY{p}{,}\PY{n}{Phi}\PY{p}{]}\PY{p}{)}\PY{p}{)}
\PY{c}{\PYZsh{} RHS for equality constraints }
\PY{n}{b}\PY{o}{=}\PY{n}{matrx}\PY{p}{(}\PY{n}{y}\PY{p}{)}

\PY{n}{sol} \PY{o}{=} \PY{n}{solvers}\PY{o}{.}\PY{n}{lp}\PY{p}{(}\PY{n}{c}\PY{p}{,} \PY{n}{G}\PY{p}{,} \PY{n}{h}\PY{p}{,}\PY{n}{A}\PY{p}{,}\PY{n}{b}\PY{p}{)}

\PY{c}{\PYZsh{}nonzero entries}
\PY{n}{nze}\PY{o}{=} \PY{n}{array}\PY{p}{(}\PY{n}{sol}\PY{p}{[}\PY{l+s}{\PYZsq{}}\PY{l+s}{x}\PY{l+s}{\PYZsq{}}\PY{p}{]}\PY{p}{)}\PY{o}{.}\PY{n}{flatten}\PY{p}{(}\PY{p}{)}\PY{p}{[}\PY{p}{:}\PY{n}{Nf}\PY{p}{]}\PY{o}{.}\PY{n}{round}\PY{p}{(}\PY{l+m+mi}{2}\PY{p}{)}\PY{o}{.}\PY{n}{nonzero}\PY{p}{(}\PY{p}{)}
\PY{k}{print} \PY{n}{array}\PY{p}{(}\PY{n}{sol}\PY{p}{[}\PY{l+s}{\PYZsq{}}\PY{l+s}{x}\PY{l+s}{\PYZsq{}}\PY{p}{]}\PY{p}{)}\PY{p}{[}\PY{n}{nze}\PY{p}{]}
\end{Verbatim}

            
                \vspace{-0.2\baselineskip}
            
        \end{ColorVerbatim}
    

    

        % If the first block is an image, minipage the image.  Else
        % request a certain amount of space for the input text.
        \needspace{4\baselineskip}
        
        

            % Add document contents.
            
                \begin{InvisibleVerbatim}
                \vspace{-0.5\baselineskip}
    \begin{alltt}     pcost       dcost       gap    pres   dres   k/t
 0:  0.0000e+00 -0.0000e+00  1e+02  2e+01  1e-16  1e+00
 1:  1.6712e-01  1.6700e-01  1e+01  1e+00  2e-16  7e-02
 2:  1.2947e+00  1.2929e+00  4e+00  5e-01  3e-16  3e-02
 3:  1.3785e+00  1.3745e+00  2e+00  2e-01  6e-16  8e-03
 4:  1.4705e+00  1.4690e+00  5e-01  7e-02  4e-16  2e-03
 5:  1.4976e+00  1.4972e+00  2e-01  2e-02  6e-16  7e-04
 6:  1.4979e+00  1.4978e+00  6e-02  7e-03  3e-14  2e-04
 7:  1.4998e+00  1.4998e+00  6e-03  8e-04  2e-14  2e-05
 8:  1.5000e+00  1.5000e+00  6e-05  8e-06  3e-14  3e-07
 9:  1.5000e+00  1.5000e+00  6e-07  8e-08  2e-14  3e-09
Optimal solution found.
[[ 0.99999789]
 [ 0.49999879]]
\end{alltt}

            \end{InvisibleVerbatim}
            
        
    
That worked out! However, if you play around with this example enough
with different random matrices (unset the \texttt{seed} statement
above), you will find that it does not \emph{always} find the correct
answer. This is because the guarantees about reconstruction are all
stated probabalistically (i.e. ``high-probability''). This is another
major difference between this and Shannon sampling.

Let's encapulate the above $L_1$ minimization code so we can use it
later.


    % Make sure that atleast 4 lines are below the HR
    \needspace{4\baselineskip}

    
        \vspace{6pt}
        \makebox[0.1\linewidth]{\smaller\hfill\tt\color{nbframe-in-prompt}In\hspace{4pt}{[}5{]}:\hspace{4pt}}\\*
        \vspace{-2.65\baselineskip}
        \begin{ColorVerbatim}
            \vspace{-0.7\baselineskip}
            \begin{Verbatim}[commandchars=\\\{\}]
\PY{k+kn}{from} \PY{n+nn}{cStringIO} \PY{k+kn}{import} \PY{n}{StringIO}
\PY{k+kn}{import} \PY{n+nn}{sys}

\PY{k}{def} \PY{n+nf}{L1\PYZus{}min}\PY{p}{(}\PY{n}{Phi}\PY{p}{,}\PY{n}{y}\PY{p}{,}\PY{n}{K}\PY{p}{)}\PY{p}{:}
    \PY{c}{\PYZsh{} inequalities matrix with }
    \PY{n}{M}\PY{p}{,}\PY{n}{Nf} \PY{o}{=} \PY{n}{Phi}\PY{o}{.}\PY{n}{shape}
    \PY{n}{G}\PY{o}{=}\PY{n}{matrx}\PY{p}{(}\PY{n}{rearrange\PYZus{}G}\PY{p}{(}\PY{n}{scipy}\PY{o}{.}\PY{n}{linalg}\PY{o}{.}\PY{n}{block\PYZus{}diag}\PY{p}{(}\PY{o}{*}\PY{p}{[}\PY{n}{matrix}\PY{p}{(}\PY{p}{[}\PY{p}{[}\PY{o}{\PYZhy{}}\PY{l+m+mi}{1}\PY{p}{,}\PY{o}{\PYZhy{}}\PY{l+m+mi}{1}\PY{p}{]}\PY{p}{,}\PY{p}{[}\PY{l+m+mi}{1}\PY{p}{,}\PY{o}{\PYZhy{}}\PY{l+m+mf}{1.0}\PY{p}{]}\PY{p}{]}\PY{p}{)}\PY{p}{,}\PY{p}{]}\PY{o}{*}\PY{n}{Nf}\PY{p}{)} \PY{p}{)}\PY{p}{)}
    \PY{c}{\PYZsh{} objective function row\PYZhy{}matrix}
    \PY{n}{c}\PY{o}{=}\PY{n}{matrx}\PY{p}{(}\PY{n}{hstack}\PY{p}{(}\PY{p}{[}\PY{n}{ones}\PY{p}{(}\PY{n}{Nf}\PY{p}{)}\PY{p}{,}\PY{n}{zeros}\PY{p}{(}\PY{n}{Nf}\PY{p}{)}\PY{p}{]}\PY{p}{)}\PY{p}{)}
    \PY{c}{\PYZsh{} RHS for inequalities}
    \PY{n}{h} \PY{o}{=} \PY{n}{matrx}\PY{p}{(}\PY{p}{[}\PY{l+m+mf}{0.0}\PY{p}{,}\PY{p}{]}\PY{o}{*}\PY{p}{(}\PY{n}{Nf}\PY{o}{*}\PY{l+m+mi}{2}\PY{p}{)}\PY{p}{,}\PY{p}{(}\PY{n}{Nf}\PY{o}{*}\PY{l+m+mi}{2}\PY{p}{,}\PY{l+m+mi}{1}\PY{p}{)}\PY{p}{,}\PY{l+s}{\PYZsq{}}\PY{l+s}{d}\PY{l+s}{\PYZsq{}}\PY{p}{)} 
    \PY{c}{\PYZsh{} equality constraint matrix}
    \PY{n}{A} \PY{o}{=} \PY{n}{matrx}\PY{p}{(}\PY{n}{hstack}\PY{p}{(}\PY{p}{[}\PY{n}{Phi}\PY{o}{*}\PY{l+m+mi}{0}\PY{p}{,}\PY{n}{Phi}\PY{p}{]}\PY{p}{)}\PY{p}{)}
    \PY{c}{\PYZsh{} RHS for equality constraints }
    \PY{n}{b}\PY{o}{=}\PY{n}{matrx}\PY{p}{(}\PY{n}{y}\PY{p}{)}
    \PY{c}{\PYZsh{} suppress standard output}
    \PY{n}{old\PYZus{}stdout} \PY{o}{=} \PY{n}{sys}\PY{o}{.}\PY{n}{stdout}
    \PY{n}{sys}\PY{o}{.}\PY{n}{stdout} \PY{o}{=} \PY{n}{mystdout} \PY{o}{=} \PY{n}{StringIO}\PY{p}{(}\PY{p}{)}
    \PY{n}{sol} \PY{o}{=} \PY{n}{solvers}\PY{o}{.}\PY{n}{lp}\PY{p}{(}\PY{n}{c}\PY{p}{,} \PY{n}{G}\PY{p}{,} \PY{n}{h}\PY{p}{,}\PY{n}{A}\PY{p}{,}\PY{n}{b}\PY{p}{)}
    \PY{c}{\PYZsh{} restore standard output}
    \PY{n}{sys}\PY{o}{.}\PY{n}{stdout} \PY{o}{=} \PY{n}{old\PYZus{}stdout}
    \PY{n}{sln} \PY{o}{=} \PY{n}{array}\PY{p}{(}\PY{n}{sol}\PY{p}{[}\PY{l+s}{\PYZsq{}}\PY{l+s}{x}\PY{l+s}{\PYZsq{}}\PY{p}{]}\PY{p}{)}\PY{o}{.}\PY{n}{flatten}\PY{p}{(}\PY{p}{)}\PY{p}{[}\PY{p}{:}\PY{n}{Nf}\PY{p}{]}\PY{o}{.}\PY{n}{round}\PY{p}{(}\PY{l+m+mi}{4}\PY{p}{)}
    \PY{k}{return} \PY{n}{sln}
\end{Verbatim}

            
                \vspace{-0.2\baselineskip}
            
        \end{ColorVerbatim}
    
\section{Example: Sparse Fourier Transform
}As an additional example, let us consider the Fourier transform and see
if we can recover the sparse Fourier transform from a small set of
measurements. For simplicity, we will assume that the time domain signal
is real which automatically means that the Fourier transform is
symmetric.


    % Make sure that atleast 4 lines are below the HR
    \needspace{4\baselineskip}

    
        \vspace{6pt}
        \makebox[0.1\linewidth]{\smaller\hfill\tt\color{nbframe-in-prompt}In\hspace{4pt}{[}141{]}:\hspace{4pt}}\\*
        \vspace{-2.65\baselineskip}
        \begin{ColorVerbatim}
            \vspace{-0.7\baselineskip}
            \begin{Verbatim}[commandchars=\\\{\}]
\PY{k}{def} \PY{n+nf}{dftmatrix}\PY{p}{(}\PY{n}{N}\PY{o}{=}\PY{l+m+mi}{8}\PY{p}{)}\PY{p}{:} 
    \PY{l+s}{\PYZsq{}}\PY{l+s}{compute inverse DFT matrices}\PY{l+s}{\PYZsq{}}
    \PY{n}{n} \PY{o}{=} \PY{n}{arange}\PY{p}{(}\PY{n}{N}\PY{p}{)}
    \PY{n}{U}\PY{o}{=}\PY{n}{matrix}\PY{p}{(} \PY{n}{exp}\PY{p}{(}\PY{l+m+mi}{1j}\PY{o}{*}\PY{l+m+mi}{2}\PY{o}{*}\PY{n}{pi}\PY{o}{/}\PY{n}{N}\PY{o}{*}\PY{n}{n}\PY{o}{*}\PY{n}{n}\PY{p}{[}\PY{p}{:}\PY{p}{,}\PY{n+nb+bp}{None}\PY{p}{]} \PY{p}{)}\PY{p}{)}\PY{o}{/}\PY{n}{sqrt}\PY{p}{(}\PY{n}{N}\PY{p}{)}
    \PY{k}{return} \PY{n}{matrix}\PY{p}{(}\PY{n}{U}\PY{p}{)}

\PY{n}{Nf}\PY{o}{=}\PY{l+m+mi}{128}
\PY{n}{K}\PY{o}{=}\PY{l+m+mi}{3} \PY{c}{\PYZsh{} components}
\PY{n}{M} \PY{o}{=} \PY{l+m+mi}{8} \PY{c}{\PYZsh{} \PYZgt{} K log2(Nf/K); num of measurements}
\PY{n}{s}\PY{o}{=}\PY{n}{zeros}\PY{p}{(}\PY{p}{(}\PY{n}{Nf}\PY{p}{,}\PY{l+m+mi}{1}\PY{p}{)}\PY{p}{)} \PY{c}{\PYZsh{} sparse vector we want to find}
\PY{n}{s}\PY{p}{[}\PY{l+m+mi}{0}\PY{p}{]}\PY{o}{=}\PY{l+m+mi}{1} \PY{c}{\PYZsh{} set the K nonzero entries}
\PY{n}{s}\PY{p}{[}\PY{l+m+mi}{1}\PY{p}{]}\PY{o}{=}\PY{l+m+mf}{0.5}
\PY{n}{s}\PY{p}{[}\PY{o}{\PYZhy{}}\PY{l+m+mi}{1}\PY{p}{]} \PY{o}{=} \PY{l+m+mf}{0.5} \PY{c}{\PYZsh{} symmetric to keep inverse Fourier transform real}
\PY{n}{Phi} \PY{o}{=} \PY{n}{dftmatrix}\PY{p}{(}\PY{n}{Nf}\PY{p}{)}\PY{p}{[}\PY{p}{:}\PY{n}{M}\PY{p}{,}\PY{p}{:}\PY{p}{]} \PY{c}{\PYZsh{} take M\PYZhy{}rows}
\PY{n}{y}\PY{o}{=}\PY{n}{Phi}\PY{o}{*}\PY{n}{s} \PY{c}{\PYZsh{} measurements}
\PY{c}{\PYZsh{} have to assert the type here on my hardware}

\PY{n}{sol}\PY{o}{=}\PY{n}{L1\PYZus{}min}\PY{p}{(}\PY{n}{Phi}\PY{o}{.}\PY{n}{real}\PY{p}{,}\PY{n}{y}\PY{o}{.}\PY{n}{real}\PY{o}{.}\PY{n}{astype}\PY{p}{(}\PY{n}{np}\PY{o}{.}\PY{n}{float64}\PY{p}{)}\PY{p}{,}\PY{n}{K}\PY{p}{)}

\PY{k}{print} \PY{n}{np}\PY{o}{.}\PY{n}{allclose}\PY{p}{(}\PY{n}{s}\PY{o}{.}\PY{n}{flatten}\PY{p}{(}\PY{p}{)}\PY{p}{,}\PY{n}{sol}\PY{p}{)}
\end{Verbatim}

            
                \vspace{-0.2\baselineskip}
            
        \end{ColorVerbatim}
    

    

        % If the first block is an image, minipage the image.  Else
        % request a certain amount of space for the input text.
        \needspace{4\baselineskip}
        
        

            % Add document contents.
            
                \begin{InvisibleVerbatim}
                \vspace{-0.5\baselineskip}
    \begin{alltt}True
\end{alltt}

            \end{InvisibleVerbatim}
            
        
    


    % Make sure that atleast 4 lines are below the HR
    \needspace{4\baselineskip}

    
        \vspace{6pt}
        \makebox[0.1\linewidth]{\smaller\hfill\tt\color{nbframe-in-prompt}In\hspace{4pt}{[}140{]}:\hspace{4pt}}\\*
        \vspace{-2.65\baselineskip}
        \begin{ColorVerbatim}
            \vspace{-0.7\baselineskip}
            \begin{Verbatim}[commandchars=\\\{\}]
\PY{n}{plot}\PY{p}{(}\PY{n}{sol}\PY{p}{)}
\PY{n}{plot}\PY{p}{(}\PY{n}{y}\PY{o}{.}\PY{n}{real}\PY{p}{)}
\end{Verbatim}

            
                \vspace{-0.2\baselineskip}
            
        \end{ColorVerbatim}
    

    

        % If the first block is an image, minipage the image.  Else
        % request a certain amount of space for the input text.
        \needspace{4\baselineskip}
        
        

            % Add document contents.
            
                \makebox[0.1\linewidth]{\smaller\hfill\tt\color{nbframe-out-prompt}Out\hspace{4pt}{[}140{]}:\hspace{4pt}}\\*
                \vspace{-2.55\baselineskip}\begin{InvisibleVerbatim}
                \vspace{-0.5\baselineskip}
    \begin{alltt}[<matplotlib.lines.Line2D at 0x7884910>]\end{alltt}

            \end{InvisibleVerbatim}
            
                \begin{InvisibleVerbatim}
                \vspace{-0.5\baselineskip}\begin{center}
    \includegraphics[max size={\textwidth}{\textheight}]{_fig_2200.png}
    \par
    \end{center}
            \end{InvisibleVerbatim}
            
        
    
\section{Uniform Uncertainty Principle
}$\Phi$ obeys a UUP for sets of size $K$ if

\[    0.8 \frac{M}{N} ||f||_2^2 \leq || \Phi f||_2^2 \leq 1.2 \frac{M}{N} ||f||_2^2 \]

Measurements that satisfy this are defined as \emph{incoherent}. Given
that $f$ is $K$-sparse and we measure $y=\Phi f$, then we search for the
sparsest vector that explains the $y$ measurements and thus find $f$ as
follows:

%$min_f \\#\lbrace t: f(t) \ne 0 \rbrace$ where $\Phi f = y$

Note that the hash mark is the size (i.e.~cardinality) of the set. This
means that we are looking for the fewest individual points for $f$ that
satisfy the constraints. Unfortunately, this is not practically
possible, so we must use the $\mathbb{L}_1$ norm as a proxy for
sparsity.

Suppose $f$ is $K$-sparse and that $\Phi$ obeys UUP for sets of size
$4K$. Then we measure $y=\Phi f$ and then solve

$min_f ||f||_1$ where $\Phi f = y$

to recover $f$ exactly and we can use $M > K \log N$ measurements, where
the number of measurements is approximately equal to the number of
active components. Let's consider a concrete example of how this works.
\subsection{Example: Sampling Sinusoids
}Here, we sample in the time-domain, given that we know the signal is
sparse in the frequency domain.

\[ \hat{f}(\omega) = \sum_{i=1}^K \alpha_i \delta(\omega_i-\omega) \]

which means that it consists of $K$-sparse nonzero elements. Therefore,
the time domain signal is

\[ f(t) =  \sum_{i=1}^K \alpha_i e^{i \omega_i t} \]

where the $\alpha_i$ and $\omega_i$ are unknown. We want solve for these
%unknowns by taking $M \gt K \log N$ samples of $f$.
The problem we want to solve is

$min_g || \hat{g} ||_{L_1}$

subject to

$g(t_m)=f(t_m)$

The trick here is that are minimizing in the frequency-domain while the
constraints are in the time-domain. To make things easier, we will
restrict our attention to real time-domain signals $f$ and we will only
reconstruct the even-indexed time-samples from the signal. This means we
need a way of expressing the inverse Fourier Transform as a matrix of
equality constraints. The assumption of real-valued time-domain signals
implies the following symmetry in the frequency-domain:

$F(k) = F(N-k)^*$

where $F$ is the Fourier transform of $f$ and the asterisk denotes
complex conjugation and $k\in \lbrace 0,1,..N-1\rbrace$ and $N$ is the
Fourier Transform length. To make things even more tractable we will
assume the time-domain signal is even, which means real-valued Fourier
transform values.

Suppose that $\mathbf{U}_N$ is the $N$-point DFT-matrix. Note that we
always assume $N$ is even. Since we are dealing with only real-valued
signals, the transform is symmetric, so we only need half of the
spectrum computed. It turns out that the even-indexed time-domain
samples can be constructed as follows:

\[\mathbf{f_{even}} = \mathbf{U}\_\{N/2\}
\begin{bmatrix}\\\\
F(0)+F(N/2)^* \\\\
F(1)+F(N/2-1)^* \\\\
F(2)+F(N/2-2)^* \\\\
\dots \\\\
F(N/2-1)+F(1)^* 
\end{bmatrix}\]



We can further simplify this by breaking this into real (superscript
$R$) and imaginary (superscript $I$) parts and keeping only the real
part

\[\mathbf{f_{even}} = \mathbf{U}_{N/2}^R
\begin{bmatrix}\\\\
F(0)^R+F(N/2)^R \\\\
F(1)^R+F(N/2-1)^R \\\\
F(2)^R+F(N/2-2)^R \\\\
\dots \\\\
F(N/2-1)^R+F(1)^R 
\end{bmatrix}
+
\mathbf{U}^I_N
\begin{bmatrix} \\\\
-F(0)^I+F(N/2)^I  \\\\
-F(1)^I+F(N/2-1)^I  \\\\
-F(2)^I+F(N/2-2)^I  \\\\
\dots \\\\
-F(N/2-1)^I+F(1)^I 
\end{bmatrix}\]

But we are going to force all the $F(k)^I$ to be zero in our example.
Note that the second term should have a $\mathbf{U}_{N/2}$ in it instead
$\mathbf{U}_N$ but there is something wrong with the javascript parser
for that bit of TeX.

Now, let's see if we can walk through to step-by-step to make sure our
optimization can actually work. Note that we don't need the second term
on the right with the $F^I$ terms because by our construction, $F$ is
real.


    % Make sure that atleast 4 lines are below the HR
    \needspace{4\baselineskip}

    
        \vspace{6pt}
        \makebox[0.1\linewidth]{\smaller\hfill\tt\color{nbframe-in-prompt}In\hspace{4pt}{[}358{]}:\hspace{4pt}}\\*
        \vspace{-2.65\baselineskip}
        \begin{ColorVerbatim}
            \vspace{-0.7\baselineskip}
            \begin{Verbatim}[commandchars=\\\{\}]
\PY{k}{def} \PY{n+nf}{dftmatrix}\PY{p}{(}\PY{n}{N}\PY{o}{=}\PY{l+m+mi}{8}\PY{p}{)}\PY{p}{:} 
    \PY{l+s}{\PYZsq{}}\PY{l+s}{compute inverse DFT matrices}\PY{l+s}{\PYZsq{}}
    \PY{n}{n} \PY{o}{=} \PY{n}{arange}\PY{p}{(}\PY{n}{N}\PY{p}{)}
    \PY{n}{U}\PY{o}{=}\PY{n}{matrix}\PY{p}{(} \PY{n}{exp}\PY{p}{(}\PY{l+m+mi}{1j}\PY{o}{*}\PY{l+m+mi}{2}\PY{o}{*}\PY{n}{pi}\PY{o}{/}\PY{n}{N}\PY{o}{*}\PY{n}{n}\PY{o}{*}\PY{n}{n}\PY{p}{[}\PY{p}{:}\PY{p}{,}\PY{n+nb+bp}{None}\PY{p}{]} \PY{p}{)}\PY{p}{)}\PY{o}{/}\PY{n}{sqrt}\PY{p}{(}\PY{n}{N}\PY{p}{)}
    \PY{k}{return} \PY{n}{matrix}\PY{p}{(}\PY{n}{U}\PY{p}{)}

\PY{k}{def} \PY{n+nf}{Q\PYZus{}rmatrix}\PY{p}{(}\PY{n}{Nf}\PY{o}{=}\PY{l+m+mi}{8}\PY{p}{)}\PY{p}{:}
    \PY{l+s}{\PYZsq{}}\PY{l+s}{implements the reordering, adding, and stacking of the matrices above}\PY{l+s}{\PYZsq{}}
    \PY{n}{Q\PYZus{}r}\PY{o}{=}\PY{n}{matrix}\PY{p}{(}\PY{n}{hstack}\PY{p}{(}\PY{p}{[}\PY{n}{eye}\PY{p}{(}\PY{n}{Nf}\PY{o}{/}\PY{l+m+mi}{2}\PY{p}{)}\PY{p}{,}\PY{n}{eye}\PY{p}{(}\PY{n}{Nf}\PY{o}{/}\PY{l+m+mi}{2}\PY{p}{)}\PY{o}{*}\PY{l+m+mi}{0}\PY{p}{]}\PY{p}{)}
               \PY{o}{+}\PY{n}{hstack}\PY{p}{(}\PY{p}{[}\PY{n}{zeros}\PY{p}{(}\PY{p}{(}\PY{n}{Nf}\PY{o}{/}\PY{l+m+mi}{2}\PY{p}{,}\PY{l+m+mi}{1}\PY{p}{)}\PY{p}{)}\PY{p}{,}\PY{n}{fliplr}\PY{p}{(}\PY{n}{eye}\PY{p}{(}\PY{n}{Nf}\PY{o}{/}\PY{l+m+mi}{2}\PY{p}{)}\PY{p}{)}\PY{p}{,}\PY{n}{zeros}\PY{p}{(}\PY{p}{(}\PY{n}{Nf}\PY{o}{/}\PY{l+m+mi}{2}\PY{p}{,}\PY{n}{Nf}\PY{o}{/}\PY{l+m+mi}{2}\PY{o}{\PYZhy{}}\PY{l+m+mi}{1}\PY{p}{)}\PY{p}{)}\PY{p}{]}\PY{p}{)}\PY{p}{)}
    \PY{k}{return} \PY{n}{Q\PYZus{}r}

\PY{n}{Nf}\PY{o}{=}\PY{l+m+mi}{8}
\PY{n}{F}\PY{o}{=}\PY{n}{zeros}\PY{p}{(}\PY{p}{(}\PY{n}{Nf}\PY{p}{,}\PY{l+m+mi}{1}\PY{p}{)}\PY{p}{)} \PY{c}{\PYZsh{} 8\PYZhy{}point DFT}
\PY{n}{F}\PY{p}{[}\PY{l+m+mi}{0}\PY{p}{]}\PY{o}{=} \PY{l+m+mi}{1} \PY{c}{\PYZsh{} DC\PYZhy{}term, constant signal}
\PY{n}{n} \PY{o}{=} \PY{n}{arange}\PY{p}{(}\PY{n}{Nf}\PY{o}{/}\PY{l+m+mi}{2}\PY{p}{)}

\PY{n}{ft} \PY{o}{=} \PY{n}{dftmatrix}\PY{p}{(}\PY{n}{Nf}\PY{p}{)}\PY{o}{.}\PY{n}{H}\PY{o}{*}\PY{n}{F} \PY{c}{\PYZsh{} this gives the constant signal}

\PY{n}{Q\PYZus{}r}\PY{o}{=}\PY{n}{Q\PYZus{}rmatrix}\PY{p}{(}\PY{n}{Nf}\PY{p}{)}
\PY{n}{U}\PY{o}{=}\PY{n}{dftmatrix}\PY{p}{(}\PY{n}{Nf}\PY{o}{/}\PY{l+m+mi}{2}\PY{p}{)} \PY{c}{\PYZsh{}half inverse DFT matrix}
\PY{n}{feven}\PY{o}{=} \PY{n}{U}\PY{o}{.}\PY{n}{real}\PY{o}{*}\PY{n}{Q\PYZus{}r}\PY{o}{*}\PY{n}{F} \PY{c}{\PYZsh{} half the size}
\PY{k}{print} \PY{n}{np}\PY{o}{.}\PY{n}{allclose}\PY{p}{(}\PY{n}{feven}\PY{p}{,}\PY{n}{ft}\PY{p}{[}\PY{p}{:}\PY{p}{:}\PY{l+m+mi}{2}\PY{p}{]}\PY{p}{)} \PY{c}{\PYZsh{} retrieved even\PYZhy{}numbered samples}
\end{Verbatim}

            
                \vspace{-0.2\baselineskip}
            
        \end{ColorVerbatim}
    

    

        % If the first block is an image, minipage the image.  Else
        % request a certain amount of space for the input text.
        \needspace{4\baselineskip}
        
        

            % Add document contents.
            
                \begin{InvisibleVerbatim}
                \vspace{-0.5\baselineskip}
    \begin{alltt}False
\end{alltt}

            \end{InvisibleVerbatim}
            
        
    


    % Make sure that atleast 4 lines are below the HR
    \needspace{4\baselineskip}

    
        \vspace{6pt}
        \makebox[0.1\linewidth]{\smaller\hfill\tt\color{nbframe-in-prompt}In\hspace{4pt}{[}359{]}:\hspace{4pt}}\\*
        \vspace{-2.65\baselineskip}
        \begin{ColorVerbatim}
            \vspace{-0.7\baselineskip}
            \begin{Verbatim}[commandchars=\\\{\}]
\PY{c}{\PYZsh{} let\PYZsq{}s try this with another sparse frequency\PYZhy{}domain signal}
\PY{n}{F}\PY{o}{=}\PY{n}{zeros}\PY{p}{(}\PY{p}{(}\PY{n}{Nf}\PY{p}{,}\PY{l+m+mi}{1}\PY{p}{)}\PY{p}{)}  
\PY{n}{F}\PY{p}{[}\PY{l+m+mi}{1}\PY{p}{]}\PY{o}{=}\PY{l+m+mi}{1}
\PY{n}{F}\PY{p}{[}\PY{n}{Nf}\PY{o}{\PYZhy{}}\PY{l+m+mi}{1}\PY{p}{]}\PY{o}{=}\PY{l+m+mi}{1} \PY{c}{\PYZsh{} symmetric part}
\PY{n}{ft} \PY{o}{=} \PY{n}{dftmatrix}\PY{p}{(}\PY{n}{Nf}\PY{p}{)}\PY{o}{.}\PY{n}{H}\PY{o}{*}\PY{n}{F} \PY{c}{\PYZsh{} this gives the constant signal}
\PY{n}{feven}\PY{o}{=} \PY{n}{U}\PY{o}{.}\PY{n}{real}\PY{o}{*}\PY{n}{Q\PYZus{}r}\PY{o}{*}\PY{n}{F}  \PY{c}{\PYZsh{} half the size}
\PY{k}{print} \PY{n}{np}\PY{o}{.}\PY{n}{allclose}\PY{p}{(}\PY{n}{feven}\PY{p}{,}\PY{n}{ft}\PY{p}{[}\PY{p}{:}\PY{p}{:}\PY{l+m+mi}{2}\PY{p}{]}\PY{p}{)} \PY{c}{\PYZsh{} retrieved even\PYZhy{}numbered samples}

\PY{n}{plot}\PY{p}{(}\PY{n}{arange}\PY{p}{(}\PY{n}{Nf}\PY{p}{)}\PY{p}{,}\PY{n}{ft}\PY{o}{.}\PY{n}{real}\PY{p}{,}\PY{n}{arange}\PY{p}{(}\PY{n}{Nf}\PY{p}{)}\PY{p}{[}\PY{p}{:}\PY{p}{:}\PY{l+m+mi}{2}\PY{p}{]}\PY{p}{,}\PY{n}{feven}\PY{p}{,}\PY{l+s}{\PYZsq{}}\PY{l+s}{o}\PY{l+s}{\PYZsq{}}\PY{p}{)}
\PY{n}{xlabel}\PY{p}{(}\PY{l+s}{\PYZsq{}}\PY{l+s}{\PYZdl{}t\PYZdl{}}\PY{l+s}{\PYZsq{}}\PY{p}{,}\PY{n}{fontsize}\PY{o}{=}\PY{l+m+mi}{22}\PY{p}{)}
\PY{n}{ylabel}\PY{p}{(}\PY{l+s}{\PYZsq{}}\PY{l+s}{\PYZdl{}f(t)\PYZdl{}}\PY{l+s}{\PYZsq{}}\PY{p}{,}\PY{n}{fontsize}\PY{o}{=}\PY{l+m+mi}{22}\PY{p}{)}
\PY{n}{title}\PY{p}{(}\PY{l+s}{\PYZsq{}}\PY{l+s}{even\PYZhy{}numbered samples}\PY{l+s}{\PYZsq{}}\PY{p}{)}
\end{Verbatim}

            
                \vspace{-0.2\baselineskip}
            
        \end{ColorVerbatim}
    

    

        % If the first block is an image, minipage the image.  Else
        % request a certain amount of space for the input text.
        \needspace{4\baselineskip}
        
        

            % Add document contents.
            
                \begin{InvisibleVerbatim}
                \vspace{-0.5\baselineskip}
    \begin{alltt}False
\end{alltt}

            \end{InvisibleVerbatim}
            
                \makebox[0.1\linewidth]{\smaller\hfill\tt\color{nbframe-out-prompt}Out\hspace{4pt}{[}359{]}:\hspace{4pt}}\\*
                \vspace{-2.55\baselineskip}\begin{InvisibleVerbatim}
                \vspace{-0.5\baselineskip}
    \begin{alltt}<matplotlib.text.Text at 0x7205970>\end{alltt}

            \end{InvisibleVerbatim}
            
                \begin{InvisibleVerbatim}
                \vspace{-0.5\baselineskip}\begin{center}
    \includegraphics[max size={\textwidth}{\textheight}]{_fig_2900.png}
    \par
    \end{center}
            \end{InvisibleVerbatim}
            
        
    
We can use the above cell to create more complicated real signals. You
can experiment with the cell below. Just remember to impose the symmetry
condition!


    % Make sure that atleast 4 lines are below the HR
    \needspace{4\baselineskip}

    
        \vspace{6pt}
        \makebox[0.1\linewidth]{\smaller\hfill\tt\color{nbframe-in-prompt}In\hspace{4pt}{[}360{]}:\hspace{4pt}}\\*
        \vspace{-2.65\baselineskip}
        \begin{ColorVerbatim}
            \vspace{-0.7\baselineskip}
            \begin{Verbatim}[commandchars=\\\{\}]
\PY{n}{Nf}\PY{o}{=}\PY{l+m+mi}{32} \PY{c}{\PYZsh{} must be even}
\PY{n}{F}\PY{o}{=}\PY{n}{zeros}\PY{p}{(}\PY{p}{(}\PY{n}{Nf}\PY{p}{,}\PY{l+m+mi}{1}\PY{p}{)}\PY{p}{)}  

\PY{c}{\PYZsh{} set values and corresponding symmetry conditions}

\PY{n}{F}\PY{p}{[}\PY{l+m+mi}{7}\PY{p}{]}\PY{o}{=}\PY{l+m+mi}{1} 
\PY{n}{F}\PY{p}{[}\PY{l+m+mi}{12}\PY{p}{]}\PY{o}{=}\PY{l+m+mf}{0.5}
\PY{n}{F}\PY{p}{[}\PY{l+m+mi}{9}\PY{p}{]}\PY{o}{=}\PY{o}{\PYZhy{}}\PY{l+m+mf}{0.25}
\PY{n}{F}\PY{p}{[}\PY{n}{Nf}\PY{o}{\PYZhy{}}\PY{l+m+mi}{9}\PY{p}{]}\PY{o}{=}\PY{o}{\PYZhy{}}\PY{l+m+mf}{0.25}
\PY{n}{F}\PY{p}{[}\PY{n}{Nf}\PY{o}{\PYZhy{}}\PY{l+m+mi}{12}\PY{p}{]} \PY{o}{=} \PY{l+m+mf}{0.5}
\PY{n}{F}\PY{p}{[}\PY{n}{Nf}\PY{o}{\PYZhy{}}\PY{l+m+mi}{7}\PY{p}{]}\PY{o}{=}\PY{l+m+mi}{1} \PY{c}{\PYZsh{} symmetric part}

\PY{n}{Q\PYZus{}r}\PY{o}{=}\PY{n}{Q\PYZus{}rmatrix}\PY{p}{(}\PY{n}{Nf}\PY{p}{)}
\PY{n}{U}\PY{o}{=}\PY{n}{dftmatrix}\PY{p}{(}\PY{n}{Nf}\PY{o}{/}\PY{l+m+mi}{2}\PY{p}{)} \PY{c}{\PYZsh{}half inverse DFT matrix}
\PY{n}{ft} \PY{o}{=} \PY{n}{dftmatrix}\PY{p}{(}\PY{n}{Nf}\PY{p}{)}\PY{o}{.}\PY{n}{H}\PY{o}{*}\PY{n}{F} \PY{c}{\PYZsh{} this gives the constant signal}
\PY{n}{feven}\PY{o}{=} \PY{n}{U}\PY{o}{.}\PY{n}{real}\PY{o}{*}\PY{n}{Q\PYZus{}r}\PY{o}{*}\PY{n}{F}  \PY{c}{\PYZsh{} half the size}
\PY{k}{print} \PY{n}{np}\PY{o}{.}\PY{n}{allclose}\PY{p}{(}\PY{n}{feven}\PY{p}{,}\PY{n}{ft}\PY{p}{[}\PY{p}{:}\PY{p}{:}\PY{l+m+mi}{2}\PY{p}{]}\PY{p}{)} \PY{c}{\PYZsh{} retrieved even\PYZhy{}numbered samples}

\PY{n}{plot}\PY{p}{(}\PY{n}{arange}\PY{p}{(}\PY{n}{Nf}\PY{p}{)}\PY{p}{,}\PY{n}{ft}\PY{o}{.}\PY{n}{real}\PY{p}{,}\PY{n}{arange}\PY{p}{(}\PY{n}{Nf}\PY{p}{)}\PY{p}{[}\PY{p}{:}\PY{p}{:}\PY{l+m+mi}{2}\PY{p}{]}\PY{p}{,}\PY{n}{feven}\PY{p}{,}\PY{l+s}{\PYZsq{}}\PY{l+s}{o}\PY{l+s}{\PYZsq{}}\PY{p}{)}
\PY{n}{xlabel}\PY{p}{(}\PY{l+s}{\PYZsq{}}\PY{l+s}{\PYZdl{}t\PYZdl{}}\PY{l+s}{\PYZsq{}}\PY{p}{,}\PY{n}{fontsize}\PY{o}{=}\PY{l+m+mi}{22}\PY{p}{)}
\PY{n}{ylabel}\PY{p}{(}\PY{l+s}{\PYZsq{}}\PY{l+s}{\PYZdl{}f(t)\PYZdl{}}\PY{l+s}{\PYZsq{}}\PY{p}{,}\PY{n}{fontsize}\PY{o}{=}\PY{l+m+mi}{22}\PY{p}{)}
\PY{n}{title}\PY{p}{(}\PY{l+s}{\PYZsq{}}\PY{l+s}{even\PYZhy{}numbered samples}\PY{l+s}{\PYZsq{}}\PY{p}{)}
\end{Verbatim}

            
                \vspace{-0.2\baselineskip}
            
        \end{ColorVerbatim}
    

    

        % If the first block is an image, minipage the image.  Else
        % request a certain amount of space for the input text.
        \needspace{4\baselineskip}
        
        

            % Add document contents.
            
                \begin{InvisibleVerbatim}
                \vspace{-0.5\baselineskip}
    \begin{alltt}False
\end{alltt}

            \end{InvisibleVerbatim}
            
                \makebox[0.1\linewidth]{\smaller\hfill\tt\color{nbframe-out-prompt}Out\hspace{4pt}{[}360{]}:\hspace{4pt}}\\*
                \vspace{-2.55\baselineskip}\begin{InvisibleVerbatim}
                \vspace{-0.5\baselineskip}
    \begin{alltt}<matplotlib.text.Text at 0x73d8f10>\end{alltt}

            \end{InvisibleVerbatim}
            
                \begin{InvisibleVerbatim}
                \vspace{-0.5\baselineskip}\begin{center}
    \includegraphics[max size={\textwidth}{\textheight}]{_fig_3100.png}
    \par
    \end{center}
            \end{InvisibleVerbatim}
            
        
    
Now that we have gone through all that trouble to create the
even-samples matrix, we can finally put it into the framework of the
$L_1$ minimization problem:

$min_F || \mathbf{F} ||_{L_1}$

subject to

$\mathbf{U}_{N/2}^R \mathbf{Q}_r \mathbf{F}= \mathbf{f}$


    % Make sure that atleast 4 lines are below the HR
    \needspace{4\baselineskip}

    
        \vspace{6pt}
        \makebox[0.1\linewidth]{\smaller\hfill\tt\color{nbframe-in-prompt}In\hspace{4pt}{[}361{]}:\hspace{4pt}}\\*
        \vspace{-2.65\baselineskip}
        \begin{ColorVerbatim}
            \vspace{-0.7\baselineskip}
            \begin{Verbatim}[commandchars=\\\{\}]
\PY{k}{def} \PY{n+nf}{rearrange\PYZus{}G}\PY{p}{(} \PY{n}{x} \PY{p}{)}\PY{p}{:} 
    \PY{l+s}{\PYZsq{}}\PY{l+s}{setup to put inequalities matrix with first 1/2 of elements as main variables}\PY{l+s}{\PYZsq{}}
    \PY{n}{n}\PY{o}{=}\PY{n}{x}\PY{o}{.}\PY{n}{shape}\PY{p}{[}\PY{l+m+mi}{0}\PY{p}{]}
    \PY{k}{return} \PY{n}{hstack}\PY{p}{(}\PY{p}{[}\PY{n}{x}\PY{p}{[}\PY{p}{:}\PY{p}{,}\PY{n}{arange}\PY{p}{(}\PY{l+m+mi}{0}\PY{p}{,}\PY{n}{n}\PY{p}{,}\PY{l+m+mi}{2}\PY{p}{)}\PY{o}{+}\PY{l+m+mi}{1}\PY{p}{]}\PY{p}{,} \PY{n}{x}\PY{p}{[}\PY{p}{:}\PY{p}{,}\PY{n}{arange}\PY{p}{(}\PY{l+m+mi}{0}\PY{p}{,}\PY{n}{n}\PY{p}{,}\PY{l+m+mi}{2}\PY{p}{)}\PY{p}{]}\PY{p}{]}\PY{p}{)}

\PY{n}{K}\PY{o}{=}\PY{l+m+mi}{2} \PY{c}{\PYZsh{} components}
\PY{n}{Nf}\PY{o}{=}\PY{l+m+mi}{128} \PY{c}{\PYZsh{} number of samples}
\PY{n}{M} \PY{o}{=} \PY{l+m+mi}{18} \PY{c}{\PYZsh{} \PYZgt{} K log(N); num of measurements}

\PY{c}{\PYZsh{} setup signal DFT as F}
\PY{n}{F}\PY{o}{=}\PY{n}{zeros}\PY{p}{(}\PY{p}{(}\PY{n}{Nf}\PY{p}{,}\PY{l+m+mi}{1}\PY{p}{)}\PY{p}{)}  
\PY{n}{F}\PY{p}{[}\PY{l+m+mi}{1}\PY{p}{]}\PY{o}{=}\PY{l+m+mi}{1}
\PY{n}{F}\PY{p}{[}\PY{l+m+mi}{2}\PY{p}{]}\PY{o}{=}\PY{l+m+mf}{0.5}
\PY{n}{F}\PY{p}{[}\PY{n}{Nf}\PY{o}{\PYZhy{}}\PY{l+m+mi}{1}\PY{p}{]}\PY{o}{=}\PY{l+m+mi}{1} \PY{c}{\PYZsh{} symmetric parts}
\PY{n}{F}\PY{p}{[}\PY{n}{Nf}\PY{o}{\PYZhy{}}\PY{l+m+mi}{2}\PY{p}{]}\PY{o}{=}\PY{l+m+mf}{0.5}
\PY{n}{ftime} \PY{o}{=} \PY{n}{dftmatrix}\PY{p}{(}\PY{n}{Nf}\PY{p}{)}\PY{o}{.}\PY{n}{H}\PY{o}{*}\PY{n}{F} \PY{c}{\PYZsh{} this gives the time\PYZhy{}domain signal}
\PY{n}{ftime} \PY{o}{=} \PY{n}{ftime}\PY{o}{.}\PY{n}{real} \PY{c}{\PYZsh{} it\PYZsq{}s real anyway}
\PY{n}{time\PYZus{}samples}\PY{o}{=}\PY{p}{[}\PY{l+m+mi}{0}\PY{p}{,} \PY{l+m+mi}{2}\PY{p}{,} \PY{l+m+mi}{4}\PY{p}{,} \PY{l+m+mi}{12}\PY{p}{,} \PY{l+m+mi}{14}\PY{p}{,} \PY{l+m+mi}{16}\PY{p}{,} \PY{l+m+mi}{18}\PY{p}{,} \PY{l+m+mi}{24}\PY{p}{,} \PY{l+m+mi}{34}\PY{p}{,} \PY{l+m+mi}{36}\PY{p}{,} \PY{l+m+mi}{38}\PY{p}{,} \PY{l+m+mi}{40}\PY{p}{,} \PY{l+m+mi}{44}\PY{p}{,} \PY{l+m+mi}{46}\PY{p}{,} \PY{l+m+mi}{52}\PY{p}{,} \PY{l+m+mi}{56}\PY{p}{,} \PY{l+m+mi}{54}\PY{p}{,}\PY{l+m+mi}{62}\PY{p}{]}
\PY{n}{half\PYZus{}indexed\PYZus{}time\PYZus{}samples} \PY{o}{=} \PY{p}{(}\PY{n}{array}\PY{p}{(}\PY{n}{time\PYZus{}samples}\PY{p}{)}\PY{o}{/}\PY{l+m+mi}{2}\PY{p}{)}\PY{o}{.}\PY{n}{astype}\PY{p}{(}\PY{n+nb}{int}\PY{p}{)}
\PY{n}{Phi} \PY{o}{=} \PY{n}{dftmatrix}\PY{p}{(}\PY{n}{Nf}\PY{o}{/}\PY{l+m+mi}{2}\PY{p}{)}\PY{o}{.}\PY{n}{real}\PY{o}{*}\PY{n}{Q\PYZus{}rmatrix}\PY{p}{(}\PY{n}{Nf}\PY{p}{)}
\PY{n}{Phi\PYZus{}i} \PY{o}{=} \PY{n}{Phi}\PY{p}{[}\PY{n}{half\PYZus{}indexed\PYZus{}time\PYZus{}samples}\PY{p}{,}\PY{p}{:}\PY{p}{]}

\PY{c}{\PYZsh{} inequalities matrix with }
\PY{n}{G}\PY{o}{=}\PY{n}{matrx}\PY{p}{(}\PY{n}{rearrange\PYZus{}G}\PY{p}{(}\PY{n}{scipy}\PY{o}{.}\PY{n}{linalg}\PY{o}{.}\PY{n}{block\PYZus{}diag}\PY{p}{(}\PY{o}{*}\PY{p}{[}\PY{n}{matrix}\PY{p}{(}\PY{p}{[}\PY{p}{[}\PY{o}{\PYZhy{}}\PY{l+m+mi}{1}\PY{p}{,}\PY{o}{\PYZhy{}}\PY{l+m+mi}{1}\PY{p}{]}\PY{p}{,}\PY{p}{[}\PY{l+m+mi}{1}\PY{p}{,}\PY{o}{\PYZhy{}}\PY{l+m+mf}{1.0}\PY{p}{]}\PY{p}{]}\PY{p}{)}\PY{p}{,}\PY{p}{]}\PY{o}{*}\PY{n}{Nf}\PY{p}{)} \PY{p}{)}\PY{p}{)}
\PY{c}{\PYZsh{} objective function row\PYZhy{}matrix}
\PY{n}{c}\PY{o}{=}\PY{n}{matrx}\PY{p}{(}\PY{n}{hstack}\PY{p}{(}\PY{p}{[}\PY{n}{zeros}\PY{p}{(}\PY{n}{Nf}\PY{p}{)}\PY{p}{,}\PY{n}{ones}\PY{p}{(}\PY{n}{Nf}\PY{p}{)}\PY{p}{]}\PY{p}{)}\PY{p}{)}
\PY{c}{\PYZsh{} RHS for inequalities}
\PY{n}{h} \PY{o}{=} \PY{n}{matrx}\PY{p}{(}\PY{p}{[}\PY{l+m+mf}{0.0}\PY{p}{,}\PY{p}{]}\PY{o}{*}\PY{p}{(}\PY{n}{Nf}\PY{o}{*}\PY{l+m+mi}{2}\PY{p}{)}\PY{p}{,}\PY{p}{(}\PY{n}{Nf}\PY{o}{*}\PY{l+m+mi}{2}\PY{p}{,}\PY{l+m+mi}{1}\PY{p}{)}\PY{p}{,}\PY{l+s}{\PYZsq{}}\PY{l+s}{d}\PY{l+s}{\PYZsq{}}\PY{p}{)} 
\PY{c}{\PYZsh{} equality constraint matrix}
\PY{n}{A} \PY{o}{=} \PY{n}{matrx}\PY{p}{(}\PY{n}{hstack}\PY{p}{(}\PY{p}{[}\PY{n}{Phi\PYZus{}i}\PY{p}{,}\PY{n}{Phi\PYZus{}i}\PY{o}{*}\PY{l+m+mi}{0}\PY{p}{]}\PY{p}{)}\PY{p}{)}
\PY{c}{\PYZsh{} RHS for equality constraints }
\PY{n}{b}\PY{o}{=}\PY{n}{matrx}\PY{p}{(}\PY{n}{ftime}\PY{p}{[}\PY{n}{time\PYZus{}samples}\PY{p}{]}\PY{p}{)}

\PY{n}{sol} \PY{o}{=} \PY{n}{solvers}\PY{o}{.}\PY{n}{lp}\PY{p}{(}\PY{n}{c}\PY{p}{,} \PY{n}{G}\PY{p}{,} \PY{n}{h}\PY{p}{,}\PY{n}{A}\PY{p}{,}\PY{n}{b}\PY{p}{)}
\end{Verbatim}

            
                \vspace{-0.2\baselineskip}
            
        \end{ColorVerbatim}
    

    

        % If the first block is an image, minipage the image.  Else
        % request a certain amount of space for the input text.
        \needspace{4\baselineskip}
        
        

            % Add document contents.
            
                \begin{InvisibleVerbatim}
                \vspace{-0.5\baselineskip}
    \begin{alltt}     pcost       dcost       gap    pres   dres   k/t
 0:  0.0000e+00 -0.0000e+00  4e+02  2e+01  3e+00  1e+00
 1: -1.5648e+01 -1.2218e+01  2e+03  2e+01  3e+00  4e+00
 2: -2.3184e+03 -1.7022e+03  1e+06  8e+01  1e+01  6e+02
 3: -2.2814e+05 -1.6566e+05  1e+08  8e+01  1e+01  6e+04
 4: -2.2818e+07 -1.6568e+07  1e+10  8e+01  1e+01  6e+06
 5: -2.2818e+09 -1.6568e+09  1e+12  8e+01  1e+01  6e+08
Certificate of dual infeasibility found.
\end{alltt}

            \end{InvisibleVerbatim}
            
        
    


    % Make sure that atleast 4 lines are below the HR
    \needspace{4\baselineskip}

    
        \vspace{6pt}
        \makebox[0.1\linewidth]{\smaller\hfill\tt\color{nbframe-in-prompt}In\hspace{4pt}{[}12{]}:\hspace{4pt}}\\*
        \vspace{-2.65\baselineskip}
        \begin{ColorVerbatim}
            \vspace{-0.7\baselineskip}
            \begin{Verbatim}[commandchars=\\\{\}]
\PY{k+kn}{import} \PY{n+nn}{itertools} \PY{k+kn}{as} \PY{n+nn}{it}

\PY{k}{def} \PY{n+nf}{dftmatrix}\PY{p}{(}\PY{n}{N}\PY{o}{=}\PY{l+m+mi}{8}\PY{p}{)}\PY{p}{:} 
    \PY{l+s}{\PYZsq{}}\PY{l+s}{compute inverse DFT matrices}\PY{l+s}{\PYZsq{}}
    \PY{n}{n} \PY{o}{=} \PY{n}{arange}\PY{p}{(}\PY{n}{N}\PY{p}{)}
    \PY{n}{U}\PY{o}{=}\PY{n}{matrix}\PY{p}{(} \PY{n}{exp}\PY{p}{(}\PY{l+m+mi}{1j}\PY{o}{*}\PY{l+m+mi}{2}\PY{o}{*}\PY{n}{pi}\PY{o}{/}\PY{n}{N}\PY{o}{*}\PY{n}{n}\PY{o}{*}\PY{n}{n}\PY{p}{[}\PY{p}{:}\PY{p}{,}\PY{n+nb+bp}{None}\PY{p}{]} \PY{p}{)}\PY{p}{)}\PY{o}{/}\PY{n}{sqrt}\PY{p}{(}\PY{n}{N}\PY{p}{)}
    \PY{k}{return} \PY{n}{matrix}\PY{p}{(}\PY{n}{U}\PY{p}{)}

\PY{n}{M} \PY{o}{=} \PY{l+m+mi}{3}

\PY{n}{np}\PY{o}{.}\PY{n}{random}\PY{o}{.}\PY{n}{seed}\PY{p}{(}\PY{l+m+mi}{5489}\PY{p}{)} \PY{c}{\PYZsh{} set random seed for reproducibility}
\PY{n}{Psi}\PY{o}{=} \PY{n}{dftmatrix}\PY{p}{(}\PY{l+m+mi}{128}\PY{p}{)}
\PY{n}{Phi}\PY{o}{=} \PY{n}{randn}\PY{p}{(}\PY{n}{M}\PY{p}{,}\PY{l+m+mi}{128}\PY{p}{)}
\PY{n}{s}\PY{o}{=}\PY{n}{zeros}\PY{p}{(}\PY{p}{(}\PY{l+m+mi}{128}\PY{p}{,}\PY{l+m+mi}{1}\PY{p}{)}\PY{p}{)}
\PY{n}{s}\PY{p}{[}\PY{l+m+mi}{0}\PY{p}{]}\PY{o}{=}\PY{l+m+mi}{1}
\PY{n}{s}\PY{p}{[}\PY{l+m+mi}{10}\PY{p}{]}\PY{o}{=}\PY{l+m+mi}{1}

\PY{n}{Theta} \PY{o}{=} \PY{n}{Phi}\PY{o}{*}\PY{n}{Psi}
\PY{n}{y} \PY{o}{=} \PY{n}{Theta}\PY{o}{*}\PY{n}{s}

\PY{k}{for} \PY{n}{i} \PY{o+ow}{in} \PY{n}{it}\PY{o}{.}\PY{n}{combinations}\PY{p}{(}\PY{n+nb}{range}\PY{p}{(}\PY{l+m+mi}{128}\PY{p}{)}\PY{p}{,}\PY{l+m+mi}{2}\PY{p}{)}\PY{p}{:}
   \PY{n}{sstar}\PY{o}{=}\PY{n}{zeros}\PY{p}{(}\PY{p}{(}\PY{l+m+mi}{128}\PY{p}{,}\PY{l+m+mi}{1}\PY{p}{)}\PY{p}{)}
   \PY{n}{sstar}\PY{p}{[}\PY{n}{array}\PY{p}{(}\PY{n}{i}\PY{p}{)}\PY{p}{]}\PY{o}{=}\PY{l+m+mi}{1}
   \PY{k}{if} \PY{n}{np}\PY{o}{.}\PY{n}{allclose}\PY{p}{(}\PY{n}{Theta}\PY{o}{*}\PY{n}{sstar}\PY{p}{,}\PY{n}{y}\PY{p}{)}\PY{p}{:}
      \PY{k}{break}
\PY{k}{else}\PY{p}{:}
   \PY{k}{print} \PY{l+s}{\PYZsq{}}\PY{l+s}{no solution}\PY{l+s}{\PYZsq{}}
\end{Verbatim}

            
                \vspace{-0.2\baselineskip}
            
        \end{ColorVerbatim}
    


    % Make sure that atleast 4 lines are below the HR
    \needspace{4\baselineskip}

    
        \vspace{6pt}
        \makebox[0.1\linewidth]{\smaller\hfill\tt\color{nbframe-in-prompt}In\hspace{4pt}{[}9{]}:\hspace{4pt}}\\*
        \vspace{-2.65\baselineskip}
        \begin{ColorVerbatim}
            \vspace{-0.7\baselineskip}
            \begin{Verbatim}[commandchars=\\\{\}]
\PY{o}{\PYZpc{}}\PY{k}{qtconsole}
\end{Verbatim}

            
                \vspace{-0.2\baselineskip}
            
        \end{ColorVerbatim}
    


    % Make sure that atleast 4 lines are below the HR
    \needspace{4\baselineskip}

    
        \vspace{6pt}
        \makebox[0.1\linewidth]{\smaller\hfill\tt\color{nbframe-in-prompt}In\hspace{4pt}{[}{]}:\hspace{4pt}}\\*
        \vspace{-2.65\baselineskip}
        \begin{ColorVerbatim}
            \vspace{-0.7\baselineskip}
            \begin{Verbatim}[commandchars=\\\{\}]

\end{Verbatim}

            
                \vspace{-0.2\baselineskip}
            
        \end{ColorVerbatim}
    


        \renewcommand{\indexname}{Index}
        \printindex

    % End of document
    \end{document}


